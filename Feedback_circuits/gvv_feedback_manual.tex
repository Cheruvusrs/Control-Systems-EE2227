
\documentclass[journal,12pt,twocolumn]{IEEEtran}
%\usepackage{setspace}
\usepackage{gensymb}
%\doublespacing
%\singlespacing

%\usepackage{graphicx}
%\usepackage{amssymb}
%\usepackage{relsize}
\usepackage[cmex10]{amsmath}
%\usepackage{amsthm}
%\interdisplaylinepenalty=2500
%\savesymbol{iint}
%\usepackage{txfonts}
%\restoresymbol{TXF}{iint}
%\usepackage{wasysym}
\usepackage{amsthm}
%\usepackage{iithtlc}
\usepackage{mathrsfs}
\usepackage{txfonts}
\usepackage{stfloats}
\usepackage{bm}
\usepackage{cite}
\usepackage{cases}
\usepackage{subfig}
%\usepackage{xtab}
\usepackage{longtable}
\usepackage{multirow}
%\usepackage{algorithm}
%\usepackage{algpseudocode}
\usepackage{enumitem}
\usepackage{mathtools}
\usepackage{tikz}
\usetikzlibrary{shapes,arrows}
\usepackage{verbatim}
\usepackage[american]{circuitikz}
\usepackage{amsmath}
\usepackage{bm}
\usetikzlibrary{arrows.meta,decorations.markings}
%\usepackage{tfrupee}
%\usepackage[breaklinks=true]{hyperref}
%\usepackage{stmaryrd}
%\usepackage{tkz-euclide} % loads  TikZ and tkz-base
%\usetkzobj{all}
%\usepackage{listings}
    \usepackage{color}                                            %%
    \usepackage{array}                                            %%
    \usepackage{longtable}                                        %%
    \usepackage{calc}                                             %%
    \usepackage{multirow}                                         %%
    \usepackage{hhline}                                           %%
    \usepackage{ifthen}                                           %%
  %optionally (for landscape tables embedded in another document): %%
    \usepackage{lscape}     
\usepackage{multicol}
\usepackage{chngcntr}
%\usepackage{enumerate}

%\usepackage{wasysym}
%\newcounter{MYtempeqncnt}
\DeclareMathOperator*{\Res}{Res}
%\renewcommand{\baselinestretch}{2}
\renewcommand\thesection{\arabic{section}}
\renewcommand\thesubsection{\thesection.\arabic{subsection}}
\renewcommand\thesubsubsection{\thesubsection.\arabic{subsubsection}}

\renewcommand\thesectiondis{\arabic{section}}
\renewcommand\thesubsectiondis{\thesectiondis.\arabic{subsection}}
\renewcommand\thesubsubsectiondis{\thesubsectiondis.\arabic{subsubsection}}

% correct bad hyphenation here
\hyphenation{op-tical net-works semi-conduc-tor}
\def\inputGnumericTable{}                                 %%

%\lstset{
%language=C,
%frame=single, 
%breaklines=true,
%columns=fullflexible
%}
%\lstset{
%language=tex,
%frame=single, 
%breaklines=true
%}

\begin{document}
%


\newtheorem{theorem}{Theorem}[section]
\newtheorem{problem}{Problem}
\newtheorem{proposition}{Proposition}[section]
\newtheorem{lemma}{Lemma}[section]
\newtheorem{corollary}[theorem]{Corollary}
\newtheorem{example}{Example}[section]
\newtheorem{definition}[problem]{Definition}
%\newtheorem{thm}{Theorem}[section] 
%\newtheorem{defn}[thm]{Definition}
%\newtheorem{algorithm}{Algorithm}[section]
%\newtheorem{cor}{Corollary}
\newcommand{\BEQA}{\begin{eqnarray}}
\newcommand{\EEQA}{\end{eqnarray}}
\newcommand{\define}{\stackrel{\triangle}{=}}
\bibliographystyle{IEEEtran}
%\bibliographystyle{ieeetr}
\providecommand{\mbf}{\mathbf}
\providecommand{\pr}[1]{\ensuremath{\Pr\left(#1\right)}}
\providecommand{\qfunc}[1]{\ensuremath{Q\left(#1\right)}}
\providecommand{\sbrak}[1]{\ensuremath{{}\left[#1\right]}}
\providecommand{\lsbrak}[1]{\ensuremath{{}\left[#1\right.}}
\providecommand{\rsbrak}[1]{\ensuremath{{}\left.#1\right]}}
\providecommand{\brak}[1]{\ensuremath{\left(#1\right)}}
\providecommand{\lbrak}[1]{\ensuremath{\left(#1\right.}}
\providecommand{\rbrak}[1]{\ensuremath{\left.#1\right)}}
\providecommand{\cbrak}[1]{\ensuremath{\left\{#1\right\}}}
\providecommand{\lcbrak}[1]{\ensuremath{\left\{#1\right.}}
\providecommand{\rcbrak}[1]{\ensuremath{\left.#1\right\}}}
\theoremstyle{remark}
\newtheorem{rem}{Remark}
\newcommand{\sgn}{\mathop{\mathrm{sgn}}}
\providecommand{\abs}[1]{\left\vert#1\right\vert}
\providecommand{\res}[1]{\Res\displaylimits_{#1}} 
\providecommand{\norm}[1]{\left\lVert#1\right\rVert}
%\providecommand{\norm}[1]{\lVert#1\rVert}
\providecommand{\mtx}[1]{\mathbf{#1}}
\providecommand{\mean}[1]{E\left[ #1 \right]}
\providecommand{\fourier}{\overset{\mathcal{F}}{ \rightleftharpoons}}
%\providecommand{\hilbert}{\overset{\mathcal{H}}{ \rightleftharpoons}}
\providecommand{\system}{\overset{\mathcal{H}}{ \longleftrightarrow}}
	%\newcommand{\solution}[2]{\textbf{Solution:}{#1}}
\newcommand{\solution}{\noindent \textbf{Solution: }}
\newcommand{\cosec}{\,\text{cosec}\,}
\providecommand{\dec}[2]{\ensuremath{\overset{#1}{\underset{#2}{\gtrless}}}}
\newcommand{\myvec}[1]{\ensuremath{\begin{pmatrix}#1\end{pmatrix}}}
\newcommand{\mydet}[1]{\ensuremath{\begin{vmatrix}#1\end{vmatrix}}}
%\numberwithin{equation}{section}
\numberwithin{equation}{subsection}
%\numberwithin{problem}{section}
%\numberwithin{definition}{section}
\makeatletter
\@addtoreset{figure}{problem}
\makeatother
\let\StandardTheFigure\thefigure
\let\vec\mathbf
%\renewcommand{\thefigure}{\theproblem.\arabic{figure}}
\renewcommand{\thefigure}{\theproblem}
%\setlist[enumerate,1]{before=\renewcommand\theequation{\theenumi.\arabic{equation}}
%\counterwithin{equation}{enumi}
%\renewcommand{\theequation}{\arabic{subsection}.\arabic{equation}}
\def\putbox#1#2#3{\makebox[0in][l]{\makebox[#1][l]{}\raisebox{\baselineskip}[0in][0in]{\raisebox{#2}[0in][0in]{#3}}}}
     \def\rightbox#1{\makebox[0in][r]{#1}}
     \def\centbox#1{\makebox[0in]{#1}}
     \def\topbox#1{\raisebox{-\baselineskip}[0in][0in]{#1}}
     \def\midbox#1{\raisebox{-0.5\baselineskip}[0in][0in]{#1}}
\vspace{3cm}
\title{
	\logo{
Control Systems
	}
}
\author{ G V V Sharma$^{*}$% <-this % stops a space
	\thanks{*The author is with the Department
		of Electrical Engineering, Indian Institute of Technology, Hyderabad
		502285 India e-mail:  gadepall@iith.ac.in. All content in this manual is released under GNU GPL.  Free and open source.}
	
}	
%\title{
%	\logo{Matrix Analysis through Octave}{\begin{center}\includegraphics[scale=.24]{tlc}\end{center}}{}{HAMDSP}
%}
% paper title
% can use linebreaks \\ within to get better formatting as desired
%\title{Matrix Analysis through Octave}
%
%
% author names and IEEE memberships
% note positions of commas and nonbreaking spaces ( ~ ) LaTeX will not break
% a structure at a ~ so this keeps an author's name from being broken across
% two lines.
% use \thanks{} to gain access to the first footnote area
% a separate \thanks must be used for each paragraph as LaTeX2e's \thanks
% was not built to handle multiple paragraphs
%
%\author{<-this % stops a space
%\thanks{}}
%}
% note the % following the last \IEEEmembership and also \thanks - 
% these prevent an unwanted space from occurring between the last author name
% and the end of the author line. i.e., if you had this:
% 
% \author{....lastname \thanks{...} \thanks{...} }
%                     ^------------^------------^----Do not want these spaces!
%
% a space would be appended to the last name and could cause every name on that
% line to be shifted left slightly. This is one of those "LaTeX things". For
% instance, "\textbf{A} \textbf{B}" will typeset as "A B" not "AB". To get
% "AB" then you have to do: "\textbf{A}\textbf{B}"
% \thanks is no different in this regard, so shield the last } of each \thanks
% that ends a line with a % and do not let a space in before the next \thanks.
% Spaces after \IEEEmembership other than the last one are OK (and needed) as
% you are supposed to have spaces between the names. For what it is worth,
% this is a minor point as most people would not even notice if the said evil
% space somehow managed to creep in.
% The paper headers
%\markboth{Journal of \LaTeX\ Class Files,~Vol.~6, No.~1, January~2007}%
%{Shell \MakeLowercase{\textit{et al.}}: Bare Demo of IEEEtran.cls for Journals}
% The only time the second header will appear is for the odd numbered pages
% after the title page when using the twoside option.
% 
% *** Note that you probably will NOT want to include the author's ***
% *** name in the headers of peer review papers.                   ***
% You can use \ifCLASSOPTIONpeerreview for conditional compilation here if
% you desire.
% If you want to put a publisher's ID mark on the page you can do it like
% this:
%\IEEEpubid{0000--0000/00\$00.00~\copyright~2007 IEEE}
% Remember, if you use this you must call \IEEEpubidadjcol in the second
% column for its text to clear the IEEEpubid mark.
% make the title area
%\maketitle
\newpage
\tableofcontents
\bigskip
\renewcommand{\thefigure}{\theenumi}
\renewcommand{\thetable}{\theenumi}
%\renewcommand{\theequation}{\theenumi}
%\begin{abstract}
%%\boldmath
%In this letter, an algorithm for evaluating the exact analytical bit error rate  (BER)  for the piecewise linear (PL) combiner for  multiple relays is presented. Previous results were available only for upto three relays. The algorithm is unique in the sense that  the actual mathematical expressions, that are prohibitively large, need not be explicitly obtained. The diversity gain due to multiple relays is shown through plots of the analytical BER, well supported by simulations. 
%
%\end{abstract}
% IEEEtran.cls defaults to using nonbold math in the Abstract.
% This preserves the distinction between vectors and scalars. However,
% if the journal you are submitting to favors bold math in the abstract,
% then you can use LaTeX's standard command \boldmath at the very start
% of the abstract to achieve this. Many IEEE journals frown on math
% in the abstract anyway.
% Note that keywords are not normally used for peerreview papers.
%\begin{IEEEkeywords}
%Cooperative diversity, decode and forward, piecewise linear
%\end{IEEEkeywords}
% For peer review papers, you can put extra information on the cover
% page as needed:
% \ifCLASSOPTIONpeerreview
% \begin{center} \bfseries EDICS Category: 3-BBND \end{center}
% \fi
%
% For peerreview papers, this IEEEtran command inserts a page break and
% creates the second title. It will be ignored for other modes.
%\IEEEpeerreviewmaketitle
\begin{abstract}
This manual is an introduction to control systems based on GATE problems.Links to sample Python codes are available in the text.  
\end{abstract}
Download python codes using 
%\begin{lstlisting}
%svn co https://github.com/gadepall/school/trunk/control/code
\section{Stability}
\section{Routh Hurwitz Criterion}
\section{Compensators}
\section{Nyquist Plot}
\section{Feedback systems}
\begin{enumerate}[label=\thesubsection.\arabic*.,ref=\thesubsection.\theenumi]
\numberwithin{equation}{enumi}

\begin{figure}[!ht]
	\begin{center}
		
		\resizebox{\columnwidth}{!}{\begin{circuitikz}[american]
\draw (0,0) node[npn](npn1){Q1}
(npn1.B) -- ++(-2,0) to [open, v^>=${V}_s$,*-] ++(0,-1) to node[ground]{}++(0,-0.25)

(npn1.E)--++(0,-0.75) to [R,l_=$R_{E1}$] ++(0,-3) to node[ground]{}++(0,-0.25)
(npn1.C) -- ++(0,0.75)to [R,l_=$R_{C1}$] ++(0,2)coordinate(A) ++(3,0);
\draw (4,1.5) node[npn](npn2){Q2}
(npn1.C)-- ++(0,0.75) -- (npn2.B)
(npn2.E) to node[ground]{}++(0,-0.0001)
(npn2.C) -- ++(0,1.5)to [R,l_=$R_{C2}$]++(0,2)coordinate(B) ++(3,0);
\draw (8,3) node[npn](npn3){Q3}
(npn2.C) --++(0,0.75)-- (npn3.B)
(npn3.C) to [short,i_<=$I_c$]++(0,1)to [R,l_=$R_{C3}$]++(0,2)coordinate(C)
(npn3.E)to [short,i_=$I_o$]++(0,-3.75)coordinate(b) to [R,l_=$R_{E3}$] ++(0,-3) to node[ground]{}++(0,-0.5)
(npn1.E) ++(0,-0.75)coordinate(a) 

(b)to [R,l_=$R_F$](a)

(npn3.C) to [short,*-]++(1,0)
node[label={right:$V_o$}]{} ++(3,0)
(A) to [short,*-]++(0,0.01) node[label={right:$+V_{cc}$}]{} ++(0,0.25)
(B) to [short,*-]++(0,0.01) node[label={right:$+V_{cc}$}]{} ++(0,0.25)
(C) to [short,*-]++(0,0.01) node[label={right:$+V_{cc}$}]{} ++(0,0.25)
(npn1.E)++(0,-0.25) to [short,*-]++(0.5,0)
node[label={right:$V_f$}]{} ++(1.5,0)
(npn1.B)++(-0.75,0) to [short,*-]++(0,1)
node[label={right:$V_i$}]{} ++(2,0);
\draw (npn3.C)++(2.75,-2)node[label={right:$R_{out}$}]{}--++(0,1)--++(-1.5,0)[->];
\draw (npn3.E)++(-2,-1)node[label={above:$R_{of}$}]{}--++(1.75,0)[->];
\draw (npn1.B)++(-2.5,-2)node[label={left:$R_{if}$}]{}--++(0,1.5)--++(0.5,0)[->]
;\end{circuitikz}
}
	\end{center}
\caption{circuit1}
\label{fig:circuit1}
\end{figure}
\begin{table}[!ht]
\centering
\begin{enumerate}[label=\thesubsection.\arabic*.,ref=\thesubsection.\theenumi]
\numberwithin{equation}{enumi}

\begin{figure}[!ht]
	\begin{center}
		
		\resizebox{\columnwidth}{!}{\begin{circuitikz}[american]
\draw (0,0) node[npn](npn1){Q1}
(npn1.B) -- ++(-2,0) to [open, v^>=${V}_s$,*-] ++(0,-1) to node[ground]{}++(0,-0.25)

(npn1.E)--++(0,-0.75) to [R,l_=$R_{E1}$] ++(0,-3) to node[ground]{}++(0,-0.25)
(npn1.C) -- ++(0,0.75)to [R,l_=$R_{C1}$] ++(0,2)coordinate(A) ++(3,0);
\draw (4,1.5) node[npn](npn2){Q2}
(npn1.C)-- ++(0,0.75) -- (npn2.B)
(npn2.E) to node[ground]{}++(0,-0.0001)
(npn2.C) -- ++(0,1.5)to [R,l_=$R_{C2}$]++(0,2)coordinate(B) ++(3,0);
\draw (8,3) node[npn](npn3){Q3}
(npn2.C) --++(0,0.75)-- (npn3.B)
(npn3.C) to [short,i_<=$I_c$]++(0,1)to [R,l_=$R_{C3}$]++(0,2)coordinate(C)
(npn3.E)to [short,i_=$I_o$]++(0,-3.75)coordinate(b) to [R,l_=$R_{E3}$] ++(0,-3) to node[ground]{}++(0,-0.5)
(npn1.E) ++(0,-0.75)coordinate(a) 

(b)to [R,l_=$R_F$](a)

(npn3.C) to [short,*-]++(1,0)
node[label={right:$V_o$}]{} ++(3,0)
(A) to [short,*-]++(0,0.01) node[label={right:$+V_{cc}$}]{} ++(0,0.25)
(B) to [short,*-]++(0,0.01) node[label={right:$+V_{cc}$}]{} ++(0,0.25)
(C) to [short,*-]++(0,0.01) node[label={right:$+V_{cc}$}]{} ++(0,0.25)
(npn1.E)++(0,-0.25) to [short,*-]++(0.5,0)
node[label={right:$V_f$}]{} ++(1.5,0)
(npn1.B)++(-0.75,0) to [short,*-]++(0,1)
node[label={right:$V_i$}]{} ++(2,0);
\draw (npn3.C)++(2.75,-2)node[label={right:$R_{out}$}]{}--++(0,1)--++(-1.5,0)[->];
\draw (npn3.E)++(-2,-1)node[label={above:$R_{of}$}]{}--++(1.75,0)[->];
\draw (npn1.B)++(-2.5,-2)node[label={left:$R_{if}$}]{}--++(0,1.5)--++(0.5,0)[->]
;\end{circuitikz}
}
	\end{center}
\caption{circuit1}
\label{fig:circuit1}
\end{figure}
\begin{table}[!ht]
\centering
\begin{enumerate}[label=\thesubsection.\arabic*.,ref=\thesubsection.\theenumi]
\numberwithin{equation}{enumi}

\begin{figure}[!ht]
	\begin{center}
		
		\resizebox{\columnwidth}{!}{\input{./figs/circuit1.tex}}
	\end{center}
\caption{circuit1}
\label{fig:circuit1}
\end{figure}
\begin{table}[!ht]
\centering
\input{./tables/EE18BTECH11007.tex}
\caption{parameters}
\label{table:ee18btech11007}
\end{table}

\item

 Part of the circuit of the MC1553 Amplifier is shown in circuit1 in  fig.\ref{fig:circuit1} Assume the loop gain is large, find an approximate expression and value for the closed loop gain $A_f=\frac{I_0}{V_s}$ and  for $\frac{I_c}{V_s}$ ,use values from Table \ref{table:ee18btech11007}

 
\begin{figure}[!ht]
	\begin{center}
		
		\resizebox{\columnwidth}{!}{\input{./figs/circuit2.tex}}
	\end{center}
\caption{circuit2}
\label{fig:circuit2}
\end{figure}
\solution When GH $>>$1,
\begin{align}
    A_f =\frac{I_0}{V_s}\approx \frac{1}{H}
\end{align}
feedback factor H can be found from feedback network.The feedback network consists of resistors $R_E_1,R_F,R_E_2$
using circuit2 in  fig.\ref{fig:circuit2} we get
\begin{align}
    H=\frac{V_f}{I_0}=\frac{R_E_2}{R_E_2+R_F+R_E_1} \times R_E_1
\end{align}
\begin{align}
    =\frac{100}{100+640+100}\times 100=11.9\Omega
\end{align}
thus,
\begin{align}
    A_f\approx \frac{1}{H}
\end{align}
\begin{align}
    =\frac{1}{R_E_2}(1+\frac{R_E_2+R_F}{R_E_1})
\end{align}
\begin{align}
    =\frac{1}{11.9}=84mA/V
\end{align}
\begin{align}
\label{eq:eq1}
    \frac{I_c}{V_s}\approx\frac{I_0}{V_s}=84 mA/V
\end{align}
\item Find $\frac{V_0}{V_s}$
\\
\solution 
\begin{align}
\frac{V_0}{V_s}=\frac{-I_c R_C_3}{V_s}=-84\times0.6=-50.4V/V
\end{align}
\item use feedback analysis to find G , H , $A_f$ , $\frac{V_0}{V_s}$ , $R_i_n$ and $R_o_u_t$.for calculating $R_o_u_t$ assume $r_0$ of $Q_3$ is 25k$\Omega$
\\
\solution employing loading rules in fig.\ref{fig:circuit1},we obtain circuit3 given in fig.\ref{fig:circuit3}
 \begin{figure}[!ht]
	\begin{center}
		
		\resizebox{\columnwidth}{!}{\input{./figs/circuit3.tex}}
	\end{center}
\caption{circuit3}
\label{fig:circuit3}
\end{figure}
\begin{figure}[!ht]
	\begin{center}
		
		\resizebox{\columnwidth}{!}{\input{./figs/circuit4.tex}}
	\end{center}
\caption{circuit4}
\label{fig:circuit4}
\end{figure}
to find $G=\frac{I_0}{V_i}$ we determine the gain of first stage,this is written by inspection as-
\begin{align}
    \frac{V_c_1}{V_i}=\frac{-\alpha(R_c_1||r_\pi_2)}{r_e_1+(R_E_1||(R_F+R_E_2))}
\end{align}
using values from \ref{table:ee18btech11007}
\begin{align}
\frac{V_c_1}{V_i}=-14.92V/V     
\end{align}
Next, we determine the gain of the second stage,which can be written by inspection(noting that $V_b_2=V_c_1$)as
{\small \begin{align}
    \frac{V_c2}{V_c_1}=-g_m_2{R_c_2||(h_f_e+1)[r_e_3+(R_E_2||(R_F+R_E_1))]}
\end{align}}%
substituting ,results in 
\begin{align}
    \frac{V_c_2}{V_c_1}=-131.2 V/V
\end{align}
Finally,for the third stage we can write by inspection
{\small \begin{align}
    \frac{I_0}{V_c_2}=\frac{I_e_3}{V_b_3}=\frac{1}{r_e_3+(R_E_2||(R_F+R_E_1))}
\end{align}}%
substituing values from \ref{table:ee18btech11007} gives
\begin{align}
    \frac{I_0}{V_c_2}=10.6mA/V
\end{align}
combining the gains of the three stags results in
{\small 
\begin{align}
G=\frac{I_0}{V_i}=-14.92\times-131.2\times10.6\times10^-3=20.7A/V    
\end{align}}%

the closed loop gain $A_f$ is found from
{\small \begin{align}
    A_f=\frac{I_0}{V_s}=\frac{G}{1+GH}=\frac{20.7}{1+20.7\times11.9}=83.7mA/V
\end{align}}%
which we note is very close to the approximate value found in \eqref{eq:eq1},above
the voltage gain is found from 
\begin{align}
    \frac{V_0}{V_s}=\frac{-I_cR_c_3}{V_s}\approx\frac{-I_0R_C_3}{V_s}=-A_fR_C_3
    \
\end{align}
\begin{align}
    =-83.7\times10^{-3}\times600=-50.2V/V
\end{align}
which is also very close to the approximate value found in   \eqref{eq:eq1} above given by
\begin{align}
    R_i_n =R_if=R_i(1+GH)
\end{align}
where $R_i$ is the input resistance of the G circuit.The value of $R_i$ can be found from the circuit in fig.\ref{fig:circuit3} as follows:
{\small \begin{align}
    R_i=(h_f_e+1)(r_e_1+(R_E_1||(R_F+R_E_2)))=13.65K\Omega
\end{align} }%
\begin{align}
    R_i_f=13.65(1+20.7\times11.9)=3.38M\Omega
\end{align}
\begin{align}
    R_o_f=R_o(1+GH)
\end{align}
where $R_o$ can be determined to be 
{\small \begin{align}
    R_o=(R_E_2||(R_F+R_E_1))+r_e_3+\frac{R_C_2}{h_f_e+1}
\end{align}}%
from values in Table \ref{table:ee18btech11007}, yields $R_o = 143.9 \Omega$. The output resistance $R_o_f$ of the feedback amplifier can now be found as
{\small \begin{align}
    R_o_f=R_o(1+GH)=143.9(1+20.7\times11.9)=35.6K\Omega
\end{align}}%
$R_o_u_t$ is found by using circuit4 in fig.\ref{fig:circuit3}
{\small \begin{align}
    R_o_u_t=r_o3+[R_o_f||(r_\pi_3+R_C_2)](1+g_m_3r_o_3\frac{r_\pi_3}{r_\pi_3+R_C_2})
\end{align}}%
{\small \begin{align}
=25+[35.6||(5.625)][1+160\times25\frac{0.625}{5.625}]=2.19M\Omega \end{align}}%

thus $R_o_u_t$ is increased (from $r_o_3$) but not by (1+GH)
\item Represent this amplifier in  a control system Block Diagram
\\
\solution figure in  fig.\ref{fig:block_diagram} represents our control system
\begin{figure}[!ht]
	\begin{center}
		
		\resizebox{\columnwidth}{!}{\input{./figs/block_diagram.tex}}
	\end{center}
\caption{block diagram}
\label{fig:block_diagram}
\end{figure}



\end{enumerate}

\caption{parameters}
\label{table:ee18btech11007}
\end{table}

\item

 Part of the circuit of the MC1553 Amplifier is shown in circuit1 in  fig.\ref{fig:circuit1} Assume the loop gain is large, find an approximate expression and value for the closed loop gain $A_f=\frac{I_0}{V_s}$ and  for $\frac{I_c}{V_s}$ ,use values from Table \ref{table:ee18btech11007}

 
\begin{figure}[!ht]
	\begin{center}
		
		\resizebox{\columnwidth}{!}{\begin{circuitikz}[american ]
\draw (0,0) to [R,l_=$R_{E1}$](0,-2) to node[ground]{}++(0,-0.25) ++(6,0)
(0,0) to [R,l_=$R_F$](3,0) to [R,l_=$R_{E2}$]++(0,-2)to node[ground]{}++(0,0)
(3,0)--(5,0)to [current source,l_=$-I_o$]++(0,-2) to node[ground]{}++(0,-0.5)
(0,0)--(-2,0) to [open, v^>=${V}_f$,*-] ++(0,-1) ++(6,0)
(0,0) 
;\end{circuitikz}
}
	\end{center}
\caption{circuit2}
\label{fig:circuit2}
\end{figure}
\solution When GH $>>$1,
\begin{align}
    A_f =\frac{I_0}{V_s}\approx \frac{1}{H}
\end{align}
feedback factor H can be found from feedback network.The feedback network consists of resistors $R_E_1,R_F,R_E_2$
using circuit2 in  fig.\ref{fig:circuit2} we get
\begin{align}
    H=\frac{V_f}{I_0}=\frac{R_E_2}{R_E_2+R_F+R_E_1} \times R_E_1
\end{align}
\begin{align}
    =\frac{100}{100+640+100}\times 100=11.9\Omega
\end{align}
thus,
\begin{align}
    A_f\approx \frac{1}{H}
\end{align}
\begin{align}
    =\frac{1}{R_E_2}(1+\frac{R_E_2+R_F}{R_E_1})
\end{align}
\begin{align}
    =\frac{1}{11.9}=84mA/V
\end{align}
\begin{align}
\label{eq:eq1}
    \frac{I_c}{V_s}\approx\frac{I_0}{V_s}=84 mA/V
\end{align}
\item Find $\frac{V_0}{V_s}$
\\
\solution 
\begin{align}
\frac{V_0}{V_s}=\frac{-I_c R_C_3}{V_s}=-84\times0.6=-50.4V/V
\end{align}
\item use feedback analysis to find G , H , $A_f$ , $\frac{V_0}{V_s}$ , $R_i_n$ and $R_o_u_t$.for calculating $R_o_u_t$ assume $r_0$ of $Q_3$ is 25k$\Omega$
\\
\solution employing loading rules in fig.\ref{fig:circuit1},we obtain circuit3 given in fig.\ref{fig:circuit3}
 \begin{figure}[!ht]
	\begin{center}
		
		\resizebox{\columnwidth}{!}{\begin{circuitikz}[american]
\draw (0,0)--(-1,0)to [voltage source,l_=$V_i$]++(0,-2)to node[ground]{}++(0,-0.5) ++(6,0)
(0.75,0) node[npn](npn1){Q1}
(npn1.C)--++(0,1.5) to [R,l_=$R_{C1}$]++(0,1.5) to node[ground,rotate=180]{}++(0,0.25)
(npn1.E)-- ++(0,-2) to [R,l_=$R_{E1}$]++(0,-1.5)to node[ground]{}++(0,-0.25)
(npn1.E)++(0,-1)to [R,l_=$R_F$]++(3,0)to[R,l_=$R_{E2}$]++(0,-1.5)to node[ground]{} ++(0,-0.25)
 (4.75,1.5)node[npn](npn2){Q2}
(npn1.C)++(0,0.75)--(npn2.B)
(npn2.E)to node[ground]{}++(0,0)
(npn2.C)--++(0,1.5)to[R,l_=$R_{C2}$]++(0,1.5)to node[ground,rotate=180]{}++(0,0.5)
(8.75,3) node[npn](npn3){Q3}
(npn2.C)++(0,0.75)--(npn3.B)
(npn3.C)to[R,l_=$R_{C3}$]++(0,1.5)to node[ground,rotate=180]{}++(0,0.25)
(npn3.E)to[short,i_=$I_o$]++(0,-2)coordinate(a)to[R,l_=$R_{E2}$]++(0,-1.5)to node[ground]{}++(0,-0.25)
(a)++(0,0.25)to[R,l_=$R_F$]++(3,0)to[R,l_=$R_{E1}$]++(0,-1.5)to node[ground]{}++(0,-0.25)

 

;\end{circuitikz}
}
	\end{center}
\caption{circuit3}
\label{fig:circuit3}
\end{figure}
\begin{figure}[!ht]
	\begin{center}
		
		\resizebox{\columnwidth}{!}{\begin{circuitikz}{american}
\draw (0,0)node[npn](npn1){Q3} ++(5,0)
(npn1.B)to[R,l_=$R_C_2$]++(-2,0) -- ++(0,-1)to node[ground]++(0,-0.25)
(npn1.E)to[R,l_=$R_o_f$]++(0,-1.5)to node[ground]++(0,-0.25)
;\end{circuitikz}}
	\end{center}
\caption{circuit4}
\label{fig:circuit4}
\end{figure}
to find $G=\frac{I_0}{V_i}$ we determine the gain of first stage,this is written by inspection as-
\begin{align}
    \frac{V_c_1}{V_i}=\frac{-\alpha(R_c_1||r_\pi_2)}{r_e_1+(R_E_1||(R_F+R_E_2))}
\end{align}
using values from \ref{table:ee18btech11007}
\begin{align}
\frac{V_c_1}{V_i}=-14.92V/V     
\end{align}
Next, we determine the gain of the second stage,which can be written by inspection(noting that $V_b_2=V_c_1$)as
{\small \begin{align}
    \frac{V_c2}{V_c_1}=-g_m_2{R_c_2||(h_f_e+1)[r_e_3+(R_E_2||(R_F+R_E_1))]}
\end{align}}%
substituting ,results in 
\begin{align}
    \frac{V_c_2}{V_c_1}=-131.2 V/V
\end{align}
Finally,for the third stage we can write by inspection
{\small \begin{align}
    \frac{I_0}{V_c_2}=\frac{I_e_3}{V_b_3}=\frac{1}{r_e_3+(R_E_2||(R_F+R_E_1))}
\end{align}}%
substituing values from \ref{table:ee18btech11007} gives
\begin{align}
    \frac{I_0}{V_c_2}=10.6mA/V
\end{align}
combining the gains of the three stags results in
{\small 
\begin{align}
G=\frac{I_0}{V_i}=-14.92\times-131.2\times10.6\times10^-3=20.7A/V    
\end{align}}%

the closed loop gain $A_f$ is found from
{\small \begin{align}
    A_f=\frac{I_0}{V_s}=\frac{G}{1+GH}=\frac{20.7}{1+20.7\times11.9}=83.7mA/V
\end{align}}%
which we note is very close to the approximate value found in \eqref{eq:eq1},above
the voltage gain is found from 
\begin{align}
    \frac{V_0}{V_s}=\frac{-I_cR_c_3}{V_s}\approx\frac{-I_0R_C_3}{V_s}=-A_fR_C_3
    \
\end{align}
\begin{align}
    =-83.7\times10^{-3}\times600=-50.2V/V
\end{align}
which is also very close to the approximate value found in   \eqref{eq:eq1} above given by
\begin{align}
    R_i_n =R_if=R_i(1+GH)
\end{align}
where $R_i$ is the input resistance of the G circuit.The value of $R_i$ can be found from the circuit in fig.\ref{fig:circuit3} as follows:
{\small \begin{align}
    R_i=(h_f_e+1)(r_e_1+(R_E_1||(R_F+R_E_2)))=13.65K\Omega
\end{align} }%
\begin{align}
    R_i_f=13.65(1+20.7\times11.9)=3.38M\Omega
\end{align}
\begin{align}
    R_o_f=R_o(1+GH)
\end{align}
where $R_o$ can be determined to be 
{\small \begin{align}
    R_o=(R_E_2||(R_F+R_E_1))+r_e_3+\frac{R_C_2}{h_f_e+1}
\end{align}}%
from values in Table \ref{table:ee18btech11007}, yields $R_o = 143.9 \Omega$. The output resistance $R_o_f$ of the feedback amplifier can now be found as
{\small \begin{align}
    R_o_f=R_o(1+GH)=143.9(1+20.7\times11.9)=35.6K\Omega
\end{align}}%
$R_o_u_t$ is found by using circuit4 in fig.\ref{fig:circuit3}
{\small \begin{align}
    R_o_u_t=r_o3+[R_o_f||(r_\pi_3+R_C_2)](1+g_m_3r_o_3\frac{r_\pi_3}{r_\pi_3+R_C_2})
\end{align}}%
{\small \begin{align}
=25+[35.6||(5.625)][1+160\times25\frac{0.625}{5.625}]=2.19M\Omega \end{align}}%

thus $R_o_u_t$ is increased (from $r_o_3$) but not by (1+GH)
\item Represent this amplifier in  a control system Block Diagram
\\
\solution figure in  fig.\ref{fig:block_diagram} represents our control system
\begin{figure}[!ht]
	\begin{center}
		
		\resizebox{\columnwidth}{!}{\tikzstyle{block} = [draw, rectangle, 
    minimum height=1.25em, minimum width=2.5em]
\tikzstyle{sum} = [draw, circle, node distance=1cm]
\tikzstyle{input} = [coordinate]
\tikzstyle{output} = [coordinate]
\tikzstyle{pinstyle} = [pin edge={to-,thin,black}]


\begin{tikzpicture}[auto, node distance=2.5cm,>=latex']
   
    \node [input, name=input] {};
    \node [sum, right of=input] (sum) {};
    \node [block, right of=sum] (controller) {$G$};
    
    \node [output, right of=controller] (output) {};
    \node [block, below of=controller] (measurements) {$H$};

   \draw [draw,->] (input) -- node[pos=0.99] {$+$} node {$V_{s}$} (sum);
    \draw [->] (sum) -- node {$V_{i}$} (controller);
    \draw [->] (controller) -- node [name=y] {$I_{o}$}(output);
    \draw [->] (y) |- (measurements);
    \draw [->] (measurements) -| node[pos=0.99] {$-$} node [near end] {$V_{f}$} (sum);
\end{tikzpicture}}
	\end{center}
\caption{block diagram}
\label{fig:block_diagram}
\end{figure}



\end{enumerate}

\caption{parameters}
\label{table:ee18btech11007}
\end{table}

\item

 Part of the circuit of the MC1553 Amplifier is shown in circuit1 in  fig.\ref{fig:circuit1} Assume the loop gain is large, find an approximate expression and value for the closed loop gain $A_f=\frac{I_0}{V_s}$ and  for $\frac{I_c}{V_s}$ ,use values from Table \ref{table:ee18btech11007}

 
\begin{figure}[!ht]
	\begin{center}
		
		\resizebox{\columnwidth}{!}{\begin{circuitikz}[american ]
\draw (0,0) to [R,l_=$R_{E1}$](0,-2) to node[ground]{}++(0,-0.25) ++(6,0)
(0,0) to [R,l_=$R_F$](3,0) to [R,l_=$R_{E2}$]++(0,-2)to node[ground]{}++(0,0)
(3,0)--(5,0)to [current source,l_=$-I_o$]++(0,-2) to node[ground]{}++(0,-0.5)
(0,0)--(-2,0) to [open, v^>=${V}_f$,*-] ++(0,-1) ++(6,0)
(0,0) 
;\end{circuitikz}
}
	\end{center}
\caption{circuit2}
\label{fig:circuit2}
\end{figure}
\solution When GH $>>$1,
\begin{align}
    A_f =\frac{I_0}{V_s}\approx \frac{1}{H}
\end{align}
feedback factor H can be found from feedback network.The feedback network consists of resistors $R_E_1,R_F,R_E_2$
using circuit2 in  fig.\ref{fig:circuit2} we get
\begin{align}
    H=\frac{V_f}{I_0}=\frac{R_E_2}{R_E_2+R_F+R_E_1} \times R_E_1
\end{align}
\begin{align}
    =\frac{100}{100+640+100}\times 100=11.9\Omega
\end{align}
thus,
\begin{align}
    A_f\approx \frac{1}{H}
\end{align}
\begin{align}
    =\frac{1}{R_E_2}(1+\frac{R_E_2+R_F}{R_E_1})
\end{align}
\begin{align}
    =\frac{1}{11.9}=84mA/V
\end{align}
\begin{align}
\label{eq:eq1}
    \frac{I_c}{V_s}\approx\frac{I_0}{V_s}=84 mA/V
\end{align}
\item Find $\frac{V_0}{V_s}$
\\
\solution 
\begin{align}
\frac{V_0}{V_s}=\frac{-I_c R_C_3}{V_s}=-84\times0.6=-50.4V/V
\end{align}
\item use feedback analysis to find G , H , $A_f$ , $\frac{V_0}{V_s}$ , $R_i_n$ and $R_o_u_t$.for calculating $R_o_u_t$ assume $r_0$ of $Q_3$ is 25k$\Omega$
\\
\solution employing loading rules in fig.\ref{fig:circuit1},we obtain circuit3 given in fig.\ref{fig:circuit3}
 \begin{figure}[!ht]
	\begin{center}
		
		\resizebox{\columnwidth}{!}{\begin{circuitikz}[american]
\draw (0,0)--(-1,0)to [voltage source,l_=$V_i$]++(0,-2)to node[ground]{}++(0,-0.5) ++(6,0)
(0.75,0) node[npn](npn1){Q1}
(npn1.C)--++(0,1.5) to [R,l_=$R_{C1}$]++(0,1.5) to node[ground,rotate=180]{}++(0,0.25)
(npn1.E)-- ++(0,-2) to [R,l_=$R_{E1}$]++(0,-1.5)to node[ground]{}++(0,-0.25)
(npn1.E)++(0,-1)to [R,l_=$R_F$]++(3,0)to[R,l_=$R_{E2}$]++(0,-1.5)to node[ground]{} ++(0,-0.25)
 (4.75,1.5)node[npn](npn2){Q2}
(npn1.C)++(0,0.75)--(npn2.B)
(npn2.E)to node[ground]{}++(0,0)
(npn2.C)--++(0,1.5)to[R,l_=$R_{C2}$]++(0,1.5)to node[ground,rotate=180]{}++(0,0.5)
(8.75,3) node[npn](npn3){Q3}
(npn2.C)++(0,0.75)--(npn3.B)
(npn3.C)to[R,l_=$R_{C3}$]++(0,1.5)to node[ground,rotate=180]{}++(0,0.25)
(npn3.E)to[short,i_=$I_o$]++(0,-2)coordinate(a)to[R,l_=$R_{E2}$]++(0,-1.5)to node[ground]{}++(0,-0.25)
(a)++(0,0.25)to[R,l_=$R_F$]++(3,0)to[R,l_=$R_{E1}$]++(0,-1.5)to node[ground]{}++(0,-0.25)

 

;\end{circuitikz}
}
	\end{center}
\caption{circuit3}
\label{fig:circuit3}
\end{figure}
\begin{figure}[!ht]
	\begin{center}
		
		\resizebox{\columnwidth}{!}{\begin{circuitikz}{american}
\draw (0,0)node[npn](npn1){Q3} ++(5,0)
(npn1.B)to[R,l_=$R_C_2$]++(-2,0) -- ++(0,-1)to node[ground]++(0,-0.25)
(npn1.E)to[R,l_=$R_o_f$]++(0,-1.5)to node[ground]++(0,-0.25)
;\end{circuitikz}}
	\end{center}
\caption{circuit4}
\label{fig:circuit4}
\end{figure}
to find $G=\frac{I_0}{V_i}$ we determine the gain of first stage,this is written by inspection as-
\begin{align}
    \frac{V_c_1}{V_i}=\frac{-\alpha(R_c_1||r_\pi_2)}{r_e_1+(R_E_1||(R_F+R_E_2))}
\end{align}
using values from \ref{table:ee18btech11007}
\begin{align}
\frac{V_c_1}{V_i}=-14.92V/V     
\end{align}
Next, we determine the gain of the second stage,which can be written by inspection(noting that $V_b_2=V_c_1$)as
{\small \begin{align}
    \frac{V_c2}{V_c_1}=-g_m_2{R_c_2||(h_f_e+1)[r_e_3+(R_E_2||(R_F+R_E_1))]}
\end{align}}%
substituting ,results in 
\begin{align}
    \frac{V_c_2}{V_c_1}=-131.2 V/V
\end{align}
Finally,for the third stage we can write by inspection
{\small \begin{align}
    \frac{I_0}{V_c_2}=\frac{I_e_3}{V_b_3}=\frac{1}{r_e_3+(R_E_2||(R_F+R_E_1))}
\end{align}}%
substituing values from \ref{table:ee18btech11007} gives
\begin{align}
    \frac{I_0}{V_c_2}=10.6mA/V
\end{align}
combining the gains of the three stags results in
{\small 
\begin{align}
G=\frac{I_0}{V_i}=-14.92\times-131.2\times10.6\times10^-3=20.7A/V    
\end{align}}%

the closed loop gain $A_f$ is found from
{\small \begin{align}
    A_f=\frac{I_0}{V_s}=\frac{G}{1+GH}=\frac{20.7}{1+20.7\times11.9}=83.7mA/V
\end{align}}%
which we note is very close to the approximate value found in \eqref{eq:eq1},above
the voltage gain is found from 
\begin{align}
    \frac{V_0}{V_s}=\frac{-I_cR_c_3}{V_s}\approx\frac{-I_0R_C_3}{V_s}=-A_fR_C_3
    \
\end{align}
\begin{align}
    =-83.7\times10^{-3}\times600=-50.2V/V
\end{align}
which is also very close to the approximate value found in   \eqref{eq:eq1} above given by
\begin{align}
    R_i_n =R_if=R_i(1+GH)
\end{align}
where $R_i$ is the input resistance of the G circuit.The value of $R_i$ can be found from the circuit in fig.\ref{fig:circuit3} as follows:
{\small \begin{align}
    R_i=(h_f_e+1)(r_e_1+(R_E_1||(R_F+R_E_2)))=13.65K\Omega
\end{align} }%
\begin{align}
    R_i_f=13.65(1+20.7\times11.9)=3.38M\Omega
\end{align}
\begin{align}
    R_o_f=R_o(1+GH)
\end{align}
where $R_o$ can be determined to be 
{\small \begin{align}
    R_o=(R_E_2||(R_F+R_E_1))+r_e_3+\frac{R_C_2}{h_f_e+1}
\end{align}}%
from values in Table \ref{table:ee18btech11007}, yields $R_o = 143.9 \Omega$. The output resistance $R_o_f$ of the feedback amplifier can now be found as
{\small \begin{align}
    R_o_f=R_o(1+GH)=143.9(1+20.7\times11.9)=35.6K\Omega
\end{align}}%
$R_o_u_t$ is found by using circuit4 in fig.\ref{fig:circuit3}
{\small \begin{align}
    R_o_u_t=r_o3+[R_o_f||(r_\pi_3+R_C_2)](1+g_m_3r_o_3\frac{r_\pi_3}{r_\pi_3+R_C_2})
\end{align}}%
{\small \begin{align}
=25+[35.6||(5.625)][1+160\times25\frac{0.625}{5.625}]=2.19M\Omega \end{align}}%

thus $R_o_u_t$ is increased (from $r_o_3$) but not by (1+GH)
\item Represent this amplifier in  a control system Block Diagram
\\
\solution figure in  fig.\ref{fig:block_diagram} represents our control system
\begin{figure}[!ht]
	\begin{center}
		
		\resizebox{\columnwidth}{!}{\tikzstyle{block} = [draw, rectangle, 
    minimum height=1.25em, minimum width=2.5em]
\tikzstyle{sum} = [draw, circle, node distance=1cm]
\tikzstyle{input} = [coordinate]
\tikzstyle{output} = [coordinate]
\tikzstyle{pinstyle} = [pin edge={to-,thin,black}]


\begin{tikzpicture}[auto, node distance=2.5cm,>=latex']
   
    \node [input, name=input] {};
    \node [sum, right of=input] (sum) {};
    \node [block, right of=sum] (controller) {$G$};
    
    \node [output, right of=controller] (output) {};
    \node [block, below of=controller] (measurements) {$H$};

   \draw [draw,->] (input) -- node[pos=0.99] {$+$} node {$V_{s}$} (sum);
    \draw [->] (sum) -- node {$V_{i}$} (controller);
    \draw [->] (controller) -- node [name=y] {$I_{o}$}(output);
    \draw [->] (y) |- (measurements);
    \draw [->] (measurements) -| node[pos=0.99] {$-$} node [near end] {$V_{f}$} (sum);
\end{tikzpicture}}
	\end{center}
\caption{block diagram}
\label{fig:block_diagram}
\end{figure}



\end{enumerate}

\end{document}