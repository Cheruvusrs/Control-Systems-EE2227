\begin{enumerate}[label=\thesubsection.\arabic*.,ref=\thesubsection.\theenumi]
\numberwithin{equation}{enumi}
\begin{figure}[!ht]
	\begin{center}
		
		\resizebox{\columnwidth}{!}{\begin{circuitikz}[american]
\draw (0,0) node[npn](npn1){Q1}
(npn1.B) -- ++(-2,0) to [open, v^>=${V}_s$,*-] ++(0,-1) to node[ground]{}++(0,-0.25)

(npn1.E)--++(0,-0.75) to [R,l_=$R_{E1}$] ++(0,-3) to node[ground]{}++(0,-0.25)
(npn1.C) -- ++(0,0.75)to [R,l_=$R_{C1}$] ++(0,2)coordinate(A) ++(3,0);
\draw (4,1.5) node[npn](npn2){Q2}
(npn1.C)-- ++(0,0.75) -- (npn2.B)
(npn2.E) to node[ground]{}++(0,-0.0001)
(npn2.C) -- ++(0,1.5)to [R,l_=$R_{C2}$]++(0,2)coordinate(B) ++(3,0);
\draw (8,3) node[npn](npn3){Q3}
(npn2.C) --++(0,0.75)-- (npn3.B)
(npn3.C) to [short,i_<=$I_c$]++(0,1)to [R,l_=$R_{C3}$]++(0,2)coordinate(C)
(npn3.E)to [short,i_=$I_o$]++(0,-3.75)coordinate(b) to [R,l_=$R_{E3}$] ++(0,-3) to node[ground]{}++(0,-0.5)
(npn1.E) ++(0,-0.75)coordinate(a) 

(b)to [R,l_=$R_F$](a)

(npn3.C) to [short,*-]++(1,0)
node[label={right:$V_o$}]{} ++(3,0)
(A) to [short,*-]++(0,0.01) node[label={right:$+V_{cc}$}]{} ++(0,0.25)
(B) to [short,*-]++(0,0.01) node[label={right:$+V_{cc}$}]{} ++(0,0.25)
(C) to [short,*-]++(0,0.01) node[label={right:$+V_{cc}$}]{} ++(0,0.25)
(npn1.E)++(0,-0.25) to [short,*-]++(0.5,0)
node[label={right:$V_f$}]{} ++(1.5,0)
(npn1.B)++(-0.75,0) to [short,*-]++(0,1)
node[label={right:$V_i$}]{} ++(2,0);
\draw (npn3.C)++(2.75,-2)node[label={right:$R_{out}$}]{}--++(0,1)--++(-1.5,0)[->];
\draw (npn3.E)++(-2,-1)node[label={above:$R_{of}$}]{}--++(1.75,0)[->];
\draw (npn1.B)++(-2.5,-2)node[label={left:$R_{if}$}]{}--++(0,1.5)--++(0.5,0)[->]
;\end{circuitikz}
}
	\end{center}
\caption{circuit1}
\label{fig:circuit1}
\end{figure}

\item
 Part of the circuit of the MC1553 Amplifier is shown in circuit1 in  fig.\ref{fig:circuit1} Assume the loop gain is large, find an approximate expression and value for the closed loop gain $A_f=\frac{I_0}{V_s}$ and hence for $\frac{I_c}{V_s}$ ,take $R_E_1=100\Omega$,$R_C_1=9K\Omega$,$R_C_2=5K\Omega$,$R_F=640\Omega$,$R_E_2=100\Omega$,$R_C_3=600\Omega$,$h_f_e=100$,$r_0=\infty$,$I_C_1=0.6mA$,$I_C_2=1mA$,$I_C_3=4mA$
 
\begin{figure}[!ht]
	\begin{center}
		
		\resizebox{\columnwidth}{!}{\begin{circuitikz}[american ]
\draw (0,0) to [R,l_=$R_{E1}$](0,-2) to node[ground]{}++(0,-0.25) ++(6,0)
(0,0) to [R,l_=$R_F$](3,0) to [R,l_=$R_{E2}$]++(0,-2)to node[ground]{}++(0,0)
(3,0)--(5,0)to [current source,l_=$-I_o$]++(0,-2) to node[ground]{}++(0,-0.5)
(0,0)--(-2,0) to [open, v^>=${V}_f$,*-] ++(0,-1) ++(6,0)
(0,0) 
;\end{circuitikz}
}
	\end{center}
\caption{circuit2}
\label{fig:circuit2}
\end{figure}
\solution When GH $>>$1,
\begin{align}
    A_f =\frac{I_0}{V_s}\approx \frac{1}{H}
\end{align}
feedback factor H can be found from feedback network.The feedback network consists of resistors $R_E_1,R_F,R_E_2$
using circuit2 in  fig.\ref{fig:circuit2} we get
\begin{align}
    H=\frac{V_f}{I_0}=\frac{R_E_2}{R_E_2+R_F+R_E_1} \times R_E_1
\end{align}
\begin{align}
    =\frac{100}{100+640+100}\times 100=11.9\Omega
\end{align}
thus,
\begin{align}
    A_f\approx \frac{1}{H}
\end{align}
\begin{align}
    =\frac{1}{R_E_2}(1+\frac{R_E_2+R_F}{R_E_1})
\end{align}
\begin{align}
    =\frac{1}{11.9}=84mA/V
\end{align}
\begin{align}
\label{eq:eq1}
    \frac{I_c}{V_s}\approx\frac{I_0}{V_s}=84 mA/V
\end{align}
\item Find $\frac{V_0}{V_s}$
\\
\solution 
\begin{align}
\frac{V_0}{V_s}=\frac{-I_c R_C_3}{V_s}=-84\times0.6=-50.4V/V
\end{align}
\item use feedback analysis to find G , H , $A_f$ , $\frac{V_0}{V_s}$ , $R_i_n$ and $R_o_u_t$.for calculating $R_o_u_t$ assume $r_0$ of $Q_3$ is 25k$\Omega$
\\
\solution employing loading rules in fig.\ref{fig:circuit1},we obtain circuit3 given in fig.\ref{fig:circuit3}
 \begin{figure}[!ht]
	\begin{center}
		
		\resizebox{\columnwidth}{!}{\begin{circuitikz}[american]
\draw (0,0)--(-1,0)to [voltage source,l_=$V_i$]++(0,-2)to node[ground]{}++(0,-0.5) ++(6,0)
(0.75,0) node[npn](npn1){Q1}
(npn1.C)--++(0,1.5) to [R,l_=$R_{C1}$]++(0,1.5) to node[ground,rotate=180]{}++(0,0.25)
(npn1.E)-- ++(0,-2) to [R,l_=$R_{E1}$]++(0,-1.5)to node[ground]{}++(0,-0.25)
(npn1.E)++(0,-1)to [R,l_=$R_F$]++(3,0)to[R,l_=$R_{E2}$]++(0,-1.5)to node[ground]{} ++(0,-0.25)
 (4.75,1.5)node[npn](npn2){Q2}
(npn1.C)++(0,0.75)--(npn2.B)
(npn2.E)to node[ground]{}++(0,0)
(npn2.C)--++(0,1.5)to[R,l_=$R_{C2}$]++(0,1.5)to node[ground,rotate=180]{}++(0,0.5)
(8.75,3) node[npn](npn3){Q3}
(npn2.C)++(0,0.75)--(npn3.B)
(npn3.C)to[R,l_=$R_{C3}$]++(0,1.5)to node[ground,rotate=180]{}++(0,0.25)
(npn3.E)to[short,i_=$I_o$]++(0,-2)coordinate(a)to[R,l_=$R_{E2}$]++(0,-1.5)to node[ground]{}++(0,-0.25)
(a)++(0,0.25)to[R,l_=$R_F$]++(3,0)to[R,l_=$R_{E1}$]++(0,-1.5)to node[ground]{}++(0,-0.25)

 

;\end{circuitikz}
}
	\end{center}
\caption{circuit3}
\label{fig:circuit3}
\end{figure}
\begin{figure}[!ht]
	\begin{center}
		
		\resizebox{\columnwidth}{!}{\begin{circuitikz}{american}
\draw (0,0)node[npn](npn1){Q3} ++(5,0)
(npn1.B)to[R,l_=$R_C_2$]++(-2,0) -- ++(0,-1)to node[ground]++(0,-0.25)
(npn1.E)to[R,l_=$R_o_f$]++(0,-1.5)to node[ground]++(0,-0.25)
;\end{circuitikz}}
	\end{center}
\caption{circuit4}
\label{fig:circuit4}
\end{figure}
to find $G=\frac{I_0}{V_i}$ we determine the gain of first stage,this is written by inspection as-
\begin{align}
    \frac{V_c_1}{V_i}=\frac{-\alpha(R_c_1||r_\pi_2)}{r_e_1+(R_E_1||(R_F+R_E_2))}
\end{align}
Since $Q_1$ is biased at 0.6mA,$r_e_1=41.7\Omega$.Transistor $Q_2$ is biased at 1mA:thus $r_\pi_2=\frac{h_f_e}{g_m_2}=\frac{100}{40}=2.5K\Omega$. Substituting these values together with $\alpha_1$=0.99,$R_C_1=9K\Omega$,$R_E_1=100\Omega$,$R_F=640\Omega$,and $R_E_2=100\Omega$, results in 
\begin{align}
   \frac{V_c_1}{V_i}=-14.92V/V 
\end{align}
Next, we determine the gain of the second stage,which can be written by inspection(noting that $V_b_2=V_c_1$)as
\begin{align}
    \frac{V_c2}{V_c_1}=-g_m_2{R_c_2||(h_f_e+1)[r_e_3+(R_E_2||(R_F+R_E_1))]}
\end{align}
substituting the values ,results in 
\begin{align}
    \frac{V_c_2}{V_c_1}=-131.2 V/V
\end{align}
Finally,for the third stage we can write by inspection
\begin{align}
    \frac{I_0}{V_c_2}=\frac{I_e_3}{V_b_3}=\frac{1}{r_e_3+(R_E_2||(R_F+R_E_1))}
\end{align}
\begin{align}
    \frac{1}{6.25+(100||740)}=10.6mA/V
\end{align}
combining the gains of the three stags results in
\begin{align}
G=\frac{I_0}{V_i}=-14.92\times-131.2\times10.6\times10^-3=20.7A/V    
\end{align}
the closed loop gain $A_f$ is found from
\begin{align}
    A_f=\frac{I_0}{V_s}=\frac{G}{1+GH}=\frac{20.7}{1+20.7\times11.9}=83.7mA/V
\end{align}
which we note is very close to the approximate value found in \eqref{eq:eq1},above
the voltage gain is found from 
\begin{align}
    \frac{V_0}{V_s}=\frac{-I_cR_c_3}{V_s}\approx\frac{-I_0R_C_3}{V_s}=-A_fR_C_3
    \
\end{align}
\begin{align}
    =-83.7\times10^{-3}\times600=-50.2V/V
\end{align}
which is also very close to the approximate value found in   \eqref{eq:eq1} above given by
\begin{align}
    R_in =R_if=R_i(1+GH)
\end{align}
where $R_i$ is the input resistance of the A circuit.The value of $R_i$ can be found from the circuit in fig.\ref{fig:circuit3} as follows:
\begin{align}
    R_i=(h_f_e+1)(r_e_1+(R_E_1||(R_F+R_E_2)))=13.65K\Omega
\end{align}
\begin{align}
    R_i_f=13.65(1+20.7\times11.9)=3.38M\Omega
\end{align}

\begin{align}
    R_o_f=R_o(1+AH)
\end{align}
where $R_o$ can be determined to be 
\begin{align}
    R_o=(R_E_2||(R_F+R_E_1))+r_e_3+\frac{R_C_2}{h_f_e+1}
\end{align}
which, for the values given, yields $R_o = 143.9 \Omega$. The output resistance $R_o_f$ of the feedback amplifier can now be found as
\begin{align}
    R_o_f=R_o(1+GH)=143.9(1+20.7\times11.9)=35.6K\Omega
\end{align}

\begin{align}
    R_o_u_t=r_o3+[R_o_f||(r_\pi_3+R_C_2)](1+g_m_3r_o_3\frac{r_\pi_3}{r_\pi_3+R_C_2})
\end{align}
\begin{align}
=25+[35.6||(5.625)][1+160\times25\frac{0.625}{5.625}]=2.19M\Omega \end{align}
thus $R_o_u_t$ is increased (from $r_o_3$) but not by (1+GH)

\end{enumerate}