\begin{enumerate}[label=\thesubsection.\arabic*.,ref=\thesubsection.\theenumi]
\numberwithin{equation}{enumi}
{\small
 \item Part of the circuit of the MC1553 Amplifier is shown in circuit1 in  fig.\ref{fig:circuit1} Assume the loop gain is large, find an approximate expression and value for the closed loop gain $T=\frac{I_0}{V_s}$ use values from Table
 \begin{figure}[!ht]
	\begin{center}
		
		\resizebox{\columnwidth}{!}{\begin{circuitikz}[american]
\draw (0,0) node[npn](npn1){Q1}
(npn1.B) -- ++(-1,0) to [open, v^>=${V}_s$] ++(0,-1) to node[ground]{}++(0,-0.25)

(npn1.E) to [R,l_=$R_E_1$] ++(0,-3) to node[ground]{}++(0,-0.25)
(npn1.C) to [R,l_=$R_C_1$] ++(0,2) ++(3,0)
\draw (2,0.75) node[npn](npn2){Q2}
(npn1.C) -- (npn2.B)
(npn2.E) to node[ground]{}++(0,-0.0001)
(npn2.C) to [R,l_=$R_C_2$]++(0,2) ++(3,0)
\draw (4,1.5) node[npn](npn3){Q3}
(npn2.C) -- (npn3.B)
(npn3.C) to [R,l_=$R_C_3$]++(0,1.5) to [short,i_<=$I_c$]++(0,0.5)
(npn3.E)to [short,i_=$I_o$]++(0,-2)coordinate(b) to [R,l_=$R_E_3$] ++(0,-3) to node[ground]{}++(0,-0.5)
(npn1.E) ++(0,-0.5)coordinate(a) 

(b)to [R,l_=$R_F$](a)

(npn3.C) to [short]++(1,0)
node[label={right:$V_o$}]{} ++(3,0)



;\end{circuitikz}}
	\end{center}
\caption{circuit1}
\label{fig:circuit1}
\end{figure}
 \ref{table:ee18btech11007}
\begin{table}[!ht]
\centering
\input{./tables/table1.tex}
\caption{parameters}
\label{table:ee18btech11007}
\end{table}
\begin{figure}[!ht]
	\begin{center}
		
		\resizebox{\columnwidth}{!}{\begin{circuitikz}
\ctikzset{bipoles/length=1cm}

\draw 
(0,0) -- ++(1,0) to [R, l_=$R_E_1$]++(0,-1)to node[ground]{}++(0,0.1)
(0,0) -- ++(1.5,0) to [R, l_=$R_F$]++(1,0) -- ++(0.75,0)coordinate(a) to [R, l_=$R_E_2$] ++(0,-1) to node[ground]{}++(0,-0.1)
(a)++(1,-1)coordinate(b) to node[ground]{}++(0,-0.1)
coordinate(b) to [american current source, label=$I_{0}$]++(0,1.1)to (a)
(-0.5,0) to  [open, v^>=${V}_f$] ++(0,-1.75) 

;\end{circuitikz}}
	\end{center}
\caption{circuit2}
\label{fig:circuit2}
\end{figure}
\\
\solution When GH $>>$1,
\begin{align}
    T =\frac{I_0}{V_s}\approx \frac{1}{H}
\end{align}
feedback factor H can be found from feedback network,The feedback network consists of resistors $R_{E1},R_F,R_{E2}$ using circuit2 in  fig.\ref{fig:circuit2} we get
\begin{align}
    H=\frac{V_f}{I_0}=\frac{R_{E2}}{R_{E2}+R_F+R_{E1}} \times R_{E1}
\end{align}
\begin{align}
    =\frac{100}{100+640+100}\times 100=11.9\Omega
\end{align}
thus,
\begin{align}
    T\approx \frac{1}{H}
\end{align}
\begin{align}
    =\frac{1}{R_{E2}}(1+\frac{R_{E2}+R_F}{R_{E1}})
\end{align}
\begin{align}
    =\frac{1}{11.9}=84mA/V
\end{align}
\begin{align}
\label{eq:eq1}
    \frac{I_c}{V_s}\approx\frac{I_0}{V_s}=84 mA/V
\end{align}
\item Find Voltage gain $\frac{V_0}{V_s}$ for above approximation
\\
\solution 
\begin{align}
\frac{V_0}{V_s}=\frac{-I_c R_{C3}}{V_s}=-84\times0.6=-50.4V/V
\end{align}
\item use feedback analysis to find open loop gain G 
\\
\solution employing loading rules in fig.\ref{fig:circuit1},we obtain circuit3 given in fig.\ref{fig:circuit3}
 \begin{figure}[!ht]
	\begin{center}
		
		\resizebox{\columnwidth}{!}{\begin{circuitikz}
\ctikzset{bipoles/length=1cm}
\draw
(0,0) node[npn](npn1){Q1} 
 (npn1.E) -- ++(0,-0.5) coordinate(a)to [R, l_=$R_E_1$] ++(0,-1) to node[ground]{} ++(0,-0.2)
 (a)-- ++(0,0.3)to [R, l_=$R_F$] ++(1,0)coordinate(b)
 (b)-- ++(1,0) to [R, l_=$R_E_2$] ++(0,-1) to node[ground]{} ++(0,-0.2)
 (npn1.C)-- ++(0,0.5) to [R, l_=$R_C_1$]++(0,1)coordinate(c)
 (c) -- ++(-0.5,0) to node[ground]{}++(0,-0.5)
 (npn1.C)-- ++(0.5,0) 
 (npn1.C)-- ++(1,0) node[npn](npn2){Q2}
 (npn2.E) to node[ground]{}++(0,-0.003)
 (npn2.C)-- ++(0,0.5) to [R, l_=$R_C_2$]++(0,1)coordinate(c2)
 (c2)-- ++(-0.5,0) to node[ground]{}++(0,-0.5)
 (npn2.C) -- ++(1,0) node[npn](npn3){Q3}
 (npn3.C)-- ++(0,0.5) to [R, l_=$R_C_3$]++(0,1)coordinate(c3)
 (c3)-- ++(-0.5,0) to node[ground]{}++(0,-0.5)
 (npn3.E)-- ++(0.5,0) -- ++(0,-0.5) coordinate(a2)to [R, l_=$R_E_2$] ++(0,-1) to node[ground]{} ++(0,-0.2)
 (a2)-- ++(0,0.3)to [R, l_=$R_F$] ++(1,0)coordinate(b2)
 (b2)-- ++(1,0) to [R, l_=$R_E_1$] ++(0,-1) to node[ground]{} ++(0,-0.2)
 (npn1.B)-- ++(-0.5,0) to [american voltage source, v=$V_{i}$]++ (0,-1.5) to node[ground]++ (0,-0.1)
 

;\end{circuitikz}}
	\end{center}
\caption{circuit3}
\label{fig:circuit3}
\end{figure}
\begin{figure}[!ht]
	\begin{center}
		
		\resizebox{\columnwidth}{!}{\begin{circuitikz}{scale=0.001}
\ctikzset{bipoles/length=1cm}
\draw
(0,0) node[npn](npn1){Q3} 
 (npn1.E) -- ++(0.5,0) -- ++(0,-0.5)to [R, l_=$R_o_f$] ++(0,-1) to node[ground]{} ++(0,-0.2)
 (npn1.B) to [R, l_=$R_C_1$] ++(0,-1)coordinate(a)
 (a) -- ++(0,-0.1) to node[ground]++(0,-0.1)
;\end{circuitikz}}
	\end{center}
\caption{circuit4}
\label{fig:circuit4}
\end{figure}
to find $G=\frac{I_0}{V_i}$ we determine the gain of first stage,this is written by inspection as-
\begin{align}
    \frac{V_{c1}}{V_i}=\frac{-\alpha(R_{c1}||r_{\pi2})}{r_{e1}+(R_{E1}||(R_F+R_{E2}))}
\end{align}
using values from \ref{table:ee18btech11007}
\begin{align}
\frac{V_{c1}}{V_i}=-14.92V/V     
\end{align}
Next, we determine the gain of the second stage,which can be written by inspection(noting that $V_{b2}=V_{c1}$)as
\begin{align}
    \frac{V_{c2}}{V_{c1}}=-g_{m2}{R_{c2}||(h_{fe}+1)[r_{e3}+(R_{E2}||(R_F+R_{E1}))]}
\end{align}
substituting ,results in 
\begin{align}
    \frac{V_{c2}}{V_{c1}}=-131.2 V/V
\end{align}
Finally,for the third stage we can write by inspection
\begin{align}
    \frac{I_0}{V_{c2}}=\frac{I_{e3}}{V_{b3}}=\frac{1}{r_{e3}+(R_{E2}||(R_F+R_{E1}))}
\end{align}
substituing values from \ref{table:ee18btech11007} gives
\begin{align}
    \frac{I_0}{V_{c2}}=10.6mA/V
\end{align}
combining the gains of the three stags results in
\begin{align}
G=\frac{I_0}{V_i}=-14.92\times-131.2\times10.6\times10^-3=20.7A/V    
\end{align}
\item Find closed loop gain T and Voltage Gain $V_0/V_s$
\\ \solution
 \begin{align}
    T=\frac{I_0}{V_s}=\frac{G}{1+GH}=\frac{20.7}{1+20.7\times11.9}=83.7mA/V
\end{align}
which we note is very close to the approximate value found in \eqref{eq:eq1} and the voltage gain is found from 
\begin{align}
    \frac{V_0}{V_s}=\frac{-I_cR_{c3}}{V_s}\approx\frac{-I_0R_{C3}}{V_s}=-TR_{C3}
    \
\end{align}
\begin{align}
    =-83.7\times10^{-3}\times600=-50.2V/V
\end{align}
\item Find $R_{in}$ and $R_{out}$
\\
\solution
\begin{align}
    R_{in} =R_{if}=R_i(1+GH)
\end{align}
where $R_i$ is the input resistance of the G circuit.The value of $R_i$ can be found from the circuit in fig.\ref{fig:circuit3} as follows:
\begin{align}
    R_i=(h_{fe}+1)(r_{e1}+(R_{E1}||(R_F+R_{E2})))=13.65K\Omega
\end{align} 
\begin{align}
    R_{if}=13.65(1+20.7\times11.9)=3.38M\Omega
\end{align}
\begin{align}
    R_{of}=R_o(1+GH)
\end{align}
where $R_o$ can be determined to be 
 \begin{align}
    R_o=(R_{E2}||(R_F+R_{E1}))+r_{e3}+\frac{R_{C2}}{h_{fe}+1}
\end{align}
from values in Table \ref{table:ee18btech11007}, yields $R_o = 143.9 \Omega$. The output resistance $R_{of}$ of the feedback amplifier can now be found as
\begin{align}
    R_{of}=R_o(1+GH)=143.9(1+20.7\times11.9)=35.6K\Omega
\end{align}
$R_{out}$ is found by using circuit4 in fig.\ref{fig:circuit3}
\begin{align}
    R_{out}=r_{o3}+[R_{of}||(r_{\pi3}+R_{C2})](1+g_{m3}r_{o3}\frac{r_{\pi3}}{r_{\pi3}+R_{C2}})
\end{align}
\begin{align}
=25+[35.6||(5.625)][1+160\times25\frac{0.625}{5.625}]=2.19M\Omega \end{align}

thus $R_{out}$ is increased (from $r_{o3}$) but not by (1+GH)
\item put the obtained parameters in a table
\\
\solution 
\begin{table}[!ht]
\centering
\input{./tables/table2.tex}
\caption{parameters}
\label{table:parameters}
\end{table}
\item Represent this amplifier in  a control system Block Diagram
\\
\solution figure in  fig.\ref{fig:block_diagram} represents our control system
\begin{figure}[!ht]
	\begin{center}
		
		\resizebox{\columnwidth}{!}{\input{blockdiagram}}
	\end{center}
\caption{block diagram}
\label{fig:block_diagram}
\end{figure}
\item write a code for doing calculations and verify the values obtained in \ref{table:parameters} 
\\
\solution 
following code does all the calculations of above equations to give parameters in
\ref{table:parameters} 
\begin{lstlisting}
codes/ee18btech11007/circuit_calc.py
\end{lstlisting}
}%



\end{enumerate}

