\begin{enumerate}[label=\thesubsection.\arabic*.,ref=\thesubsection.\theenumi]
\numberwithin{equation}{enumi}
{\small
 \item Part of the circuit of the MC1553 Amplifier is shown in circuit1 in  fig.\ref{fig:circuit1} ,Answer below questions using values from Table \ref{table:ee18btech11007}
 \begin{figure}[!ht]
	\begin{center}
		
		\resizebox{\columnwidth}{!}{\begin{circuitikz}[american]
\draw (0,0) node[npn](npn1){Q1}
(npn1.B) -- ++(-2,0) to [open, v^>=${V}_s$,*-] ++(0,-1) to node[ground]{}++(0,-0.25)

(npn1.E)--++(0,-0.75) to [R,l_=$R_{E1}$] ++(0,-3) to node[ground]{}++(0,-0.25)
(npn1.C) -- ++(0,0.75)to [R,l_=$R_{C1}$] ++(0,2)coordinate(A) ++(3,0);
\draw (4,1.5) node[npn](npn2){Q2}
(npn1.C)-- ++(0,0.75) -- (npn2.B)
(npn2.E) to node[ground]{}++(0,-0.0001)
(npn2.C) -- ++(0,1.5)to [R,l_=$R_{C2}$]++(0,2)coordinate(B) ++(3,0);
\draw (8,3) node[npn](npn3){Q3}
(npn2.C) --++(0,0.75)-- (npn3.B)
(npn3.C) to [short,i_<=$I_c$]++(0,1)to [R,l_=$R_{C3}$]++(0,2)coordinate(C)
(npn3.E)to [short,i_=$I_o$]++(0,-3.75)coordinate(b) to [R,l_=$R_{E3}$] ++(0,-3) to node[ground]{}++(0,-0.5)
(npn1.E) ++(0,-0.75)coordinate(a) 

(b)to [R,l_=$R_F$](a)

(npn3.C) to [short,*-]++(1,0)
node[label={right:$V_o$}]{} ++(3,0)
(A) to [short,*-]++(0,0.01) node[label={right:$+V_{cc}$}]{} ++(0,0.25)
(B) to [short,*-]++(0,0.01) node[label={right:$+V_{cc}$}]{} ++(0,0.25)
(C) to [short,*-]++(0,0.01) node[label={right:$+V_{cc}$}]{} ++(0,0.25)
(npn1.E)++(0,-0.25) to [short,*-]++(0.5,0)
node[label={right:$V_f$}]{} ++(1.5,0)
(npn1.B)++(-0.75,0) to [short,*-]++(0,1)
node[label={right:$V_i$}]{} ++(2,0);
\draw (npn3.C)++(2.75,-2)node[label={right:$R_{out}$}]{}--++(0,1)--++(-1.5,0)[->];
\draw (npn3.E)++(-2,-1)node[label={above:$R_{of}$}]{}--++(1.75,0)[->];
\draw (npn1.B)++(-2.5,-2)node[label={left:$R_{if}$}]{}--++(0,1.5)--++(0.5,0)[->]
;\end{circuitikz}
}
	\end{center}
\caption{circuit1}
\label{fig:circuit1}
\end{figure}
 
\begin{table}[!ht]
\centering
%%%%%%%%%%%%%%%%%%%%%%%%%%%%%%%%%%%%%%%%%%%%%%%%%%%%%%%%%%%%%%%%%%%%%%
%%                                                                  %%
%%  This is the header of a LaTeX2e file exported from Gnumeric.    %%
%%                                                                  %%
%%  This file can be compiled as it stands or included in another   %%
%%  LaTeX document. The table is based on the longtable package so  %%
%%  the longtable options (headers, footers...) can be set in the   %%
%%  preamble section below (see PRAMBLE).                           %%
%%                                                                  %%
%%  To include the file in another, the following two lines must be %%
%%  in the including file:                                          %%
%%        \def\inputGnumericTable{}                                 %%
%%  at the beginning of the file and:                               %%
%%        \input{name-of-this-file.tex}                             %%
%%  where the table is to be placed. Note also that the including   %%
%%  file must use the following packages for the table to be        %%
%%  rendered correctly:                                             %%
%%    \usepackage[latin1]{inputenc}                                 %%
%%    \usepackage{color}                                            %%
%%    \usepackage{array}                                            %%
%%    \usepackage{longtable}                                        %%
%%    \usepackage{calc}                                             %%
%%    \usepackage{multirow}                                         %%
%%    \usepackage{hhline}                                           %%
%%    \usepackage{ifthen}                                           %%
%%  optionally (for landscape tables embedded in another document): %%
%%    \usepackage{lscape}                                           %%
%%                                                                  %%
%%%%%%%%%%%%%%%%%%%%%%%%%%%%%%%%%%%%%%%%%%%%%%%%%%%%%%%%%%%%%%%%%%%%%%



%%  This section checks if we are begin input into another file or  %%
%%  the file will be compiled alone. First use a macro taken from   %%
%%  the TeXbook ex 7.7 (suggestion of Han-Wen Nienhuys).            %%
\def\ifundefined#1{\expandafter\ifx\csname#1\endcsname\relax}


%%  Check for the \def token for inputed files. If it is not        %%
%%  defined, the file will be processed as a standalone and the     %%
%%  preamble will be used.                                          %%
\ifundefined{inputGnumericTable}

%%  We must be able to close or not the document at the end.        %%
	\def\gnumericTableEnd{\end{document}}


%%%%%%%%%%%%%%%%%%%%%%%%%%%%%%%%%%%%%%%%%%%%%%%%%%%%%%%%%%%%%%%%%%%%%%
%%                                                                  %%
%%  This is the PREAMBLE. Change these values to get the right      %%
%%  paper size and other niceties.                                  %%
%%                                                                  %%
%%%%%%%%%%%%%%%%%%%%%%%%%%%%%%%%%%%%%%%%%%%%%%%%%%%%%%%%%%%%%%%%%%%%%%

	\documentclass[12pt%
			  %,landscape%
                    ]{report}
       \usepackage[latin1]{inputenc}
       \usepackage{fullpage}
       \usepackage{color}
       \usepackage{array}
       \usepackage{longtable}
       \usepackage{calc}
       \usepackage{multirow}
       \usepackage{hhline}
       \usepackage{ifthen}

	\begin{document}


%%  End of the preamble for the standalone. The next section is for %%
%%  documents which are included into other LaTeX2e files.          %%
\else

%%  We are not a stand alone document. For a regular table, we will %%
%%  have no preamble and only define the closing to mean nothing.   %%
    \def\gnumericTableEnd{}

%%  If we want landscape mode in an embedded document, comment out  %%
%%  the line above and uncomment the two below. The table will      %%
%%  begin on a new page and run in landscape mode.                  %%
%       \def\gnumericTableEnd{\end{landscape}}
%       \begin{landscape}


%%  End of the else clause for this file being \input.              %%
\fi

%%%%%%%%%%%%%%%%%%%%%%%%%%%%%%%%%%%%%%%%%%%%%%%%%%%%%%%%%%%%%%%%%%%%%%
%%                                                                  %%
%%  The rest is the gnumeric table, except for the closing          %%
%%  statement. Changes below will alter the table's appearance.     %%
%%                                                                  %%
%%%%%%%%%%%%%%%%%%%%%%%%%%%%%%%%%%%%%%%%%%%%%%%%%%%%%%%%%%%%%%%%%%%%%%

\providecommand{\gnumericmathit}[1]{#1} 
%%  Uncomment the next line if you would like your numbers to be in %%
%%  italics if they are italizised in the gnumeric table.           %%
%\renewcommand{\gnumericmathit}[1]{\mathit{#1}}
\providecommand{\gnumericPB}[1]%
{\let\gnumericTemp=\\#1\let\\=\gnumericTemp\hspace{0pt}}
 \ifundefined{gnumericTableWidthDefined}
        \newlength{\gnumericTableWidth}
        \newlength{\gnumericTableWidthComplete}
        \newlength{\gnumericMultiRowLength}
        \global\def\gnumericTableWidthDefined{}
 \fi
%% The following setting protects this code from babel shorthands.  %%
 \ifthenelse{\isundefined{\languageshorthands}}{}{\languageshorthands{english}}
%%  The default table format retains the relative column widths of  %%
%%  gnumeric. They can easily be changed to c, r or l. In that case %%
%%  you may want to comment out the next line and uncomment the one %%
%%  thereafter                                                      %%
\providecommand\gnumbox{\makebox[0pt]}
%%\providecommand\gnumbox[1][]{\makebox}

%% to adjust positions in multirow situations                       %%
\setlength{\bigstrutjot}{\jot}
\setlength{\extrarowheight}{\doublerulesep}

%%  The \setlongtables command keeps column widths the same across  %%
%%  pages. Simply comment out next line for varying column widths.  %%
\setlongtables

\setlength\gnumericTableWidth{%
	83pt+%
	91pt+%
0pt}
\def\gumericNumCols{2}
\setlength\gnumericTableWidthComplete{\gnumericTableWidth+%
         \tabcolsep*\gumericNumCols*2+\arrayrulewidth*\gumericNumCols}
\ifthenelse{\lengthtest{\gnumericTableWidthComplete > \linewidth}}%
         {\def\gnumericScale{\ratio{\linewidth-%
                        \tabcolsep*\gumericNumCols*2-%
                        \arrayrulewidth*\gumericNumCols}%
{\gnumericTableWidth}}}%
{\def\gnumericScale{1}}

%%%%%%%%%%%%%%%%%%%%%%%%%%%%%%%%%%%%%%%%%%%%%%%%%%%%%%%%%%%%%%%%%%%%%%
%%                                                                  %%
%% The following are the widths of the various columns. We are      %%
%% defining them here because then they are easier to change.       %%
%% Depending on the cell formats we may use them more than once.    %%
%%                                                                  %%
%%%%%%%%%%%%%%%%%%%%%%%%%%%%%%%%%%%%%%%%%%%%%%%%%%%%%%%%%%%%%%%%%%%%%%

\ifthenelse{\isundefined{\gnumericColA}}{\newlength{\gnumericColA}}{}\settowidth{\gnumericColA}{\begin{tabular}{@{}p{83pt*\gnumericScale}@{}}x\end{tabular}}
\ifthenelse{\isundefined{\gnumericColB}}{\newlength{\gnumericColB}}{}\settowidth{\gnumericColB}{\begin{tabular}{@{}p{91pt*\gnumericScale}@{}}x\end{tabular}}

\begin{tabular}[c]{%
	b{\gnumericColA}%
	b{\gnumericColB}%
	}

%%%%%%%%%%%%%%%%%%%%%%%%%%%%%%%%%%%%%%%%%%%%%%%%%%%%%%%%%%%%%%%%%%%%%%
%%  The longtable options. (Caption, headers... see Goosens, p.124) %%
%	\caption{The Table Caption.}             \\	%
% \hline	% Across the top of the table.
%%  The rest of these options are table rows which are placed on    %%
%%  the first, last or every page. Use \multicolumn if you want.    %%

%%  Header for the first page.                                      %%
%	\multicolumn{2}{c}{The First Header} \\ \hline 
%	\multicolumn{1}{c}{colTag}	%Column 1
%	&\multicolumn{1}{c}{colTag}	\\ \hline %Last column
%	\endfirsthead

%%  The running header definition.                                  %%
%	\hline
%	\multicolumn{2}{l}{\ldots\small\slshape continued} \\ \hline
%	\multicolumn{1}{c}{colTag}	%Column 1
%	&\multicolumn{1}{c}{colTag}	\\ \hline %Last column
%	\endhead

%%  The running footer definition.                                  %%
%	\hline
%	\multicolumn{2}{r}{\small\slshape continued\ldots} \\
%	\endfoot

%%  The ending footer definition.                                   %%
%	\multicolumn{2}{c}{That's all folks} \\ \hline 
%	\endlastfoot
%%%%%%%%%%%%%%%%%%%%%%%%%%%%%%%%%%%%%%%%%%%%%%%%%%%%%%%%%%%%%%%%%%%%%%

\hhline{|-|-}
	 \multicolumn{1}{|p{\gnumericColA}|}%
	{\gnumericPB{\raggedright}\gnumbox[l]{\textbf{Parameter}}}
	&\multicolumn{1}{p{\gnumericColB}|}%
	{\gnumericPB{\raggedright}\gnumbox[l]{\textbf{Value}}}
\\
\hhline{|-|-}
	 \multicolumn{1}{|p{\gnumericColA}|}%
	{\gnumericPB{\raggedright}\gnumbox[l]{\textbf{$R_{C1}$}}}
	&\multicolumn{1}{p{\gnumericColB}|}%
	{\gnumericPB{\raggedright}\gnumbox[l]{\textbf{9k\Omega}}}
\\
\hhline{|-|-}
	 \multicolumn{1}{|p{\gnumericColA}|}%
	{\gnumericPB{\raggedright}\gnumbox[l]{\textbf{$R_{E1}$}}}
	&\multicolumn{1}{p{\gnumericColB}|}%
	{\gnumericPB{\raggedright}\gnumbox[l]{\textbf{100\Omega}}}
\\
\hhline{|-|-}
	 \multicolumn{1}{|p{\gnumericColA}|}%
	{\gnumericPB{\raggedright}\gnumbox[l]{\textbf{$R_{C2}$}}}
	&\multicolumn{1}{p{\gnumericColB}|}%
	{\gnumericPB{\raggedright}\gnumbox[l]{\textbf{5k\Omega}}}
\\
\hhline{|-|-}
	 \multicolumn{1}{|p{\gnumericColA}|}%
	{\gnumericPB{\raggedright}\gnumbox[l]{\textbf{$R_F$}}}
	&\multicolumn{1}{p{\gnumericColB}|}%
	{\gnumericPB{\raggedright}\gnumbox[l]{\textbf{640\Omega}}}
\\
\hhline{|-|-}
	 \multicolumn{1}{|p{\gnumericColA}|}%
	{\gnumericPB{\raggedright}\gnumbox[l]{\textbf{$R_{E2}$}}}
	&\multicolumn{1}{p{\gnumericColB}|}%
	{\gnumericPB{\raggedright}\gnumbox[l]{\textbf{100\Omega}}}
\\
\hhline{|-|-}
	 \multicolumn{1}{|p{\gnumericColA}|}%
	{\gnumericPB{\raggedright}\gnumbox[l]{\textbf{$R_{C3}$}}}
	&\multicolumn{1}{p{\gnumericColB}|}%
	{\gnumericPB{\raggedright}\gnumbox[l]{\textbf{600\Omega}}}
\\
\hhline{|-|-}
	 \multicolumn{1}{|p{\gnumericColA}|}%
	{\gnumericPB{\raggedright}\gnumbox[l]{\textbf{$h_{fe}$}}}
	&\multicolumn{1}{p{\gnumericColB}|}%
	{\gnumericPB{\raggedright}\gnumbox[l]{\textbf{100}}}
\\
\hhline{|-|-}
	 \multicolumn{1}{|p{\gnumericColA}|}%
	{\gnumericPB{\raggedright}\gnumbox[l]{\textbf{$r_o$}}}
	&\multicolumn{1}{p{\gnumericColB}|}%
	{\gnumericPB{\raggedright}\gnumbox[l]{\textbf{\infty\Omega}}}
\\
\hhline{|-|-}
	 \multicolumn{1}{|p{\gnumericColA}|}%
	{\gnumericPB{\raggedright}\gnumbox[l]{\textbf{$I_{C1}$}}}
	&\multicolumn{1}{p{\gnumericColB}|}%
	{\gnumericPB{\raggedright}\gnumbox[l]{\textbf{0.6mA}}}
\\
\hhline{|-|-}
	 \multicolumn{1}{|p{\gnumericColA}|}%
	{\gnumericPB{\raggedright}\gnumbox[l]{\textbf{$I_{C2}$}}}
	&\multicolumn{1}{p{\gnumericColB}|}%
	{\gnumericPB{\raggedright}\gnumbox[l]{\textbf{1mA}}}
\\
\hhline{|-|-}
	 \multicolumn{1}{|p{\gnumericColA}|}%
	{\gnumericPB{\raggedright}\gnumbox[l]{\textbf{$I_{C3}$}}}
	&\multicolumn{1}{p{\gnumericColB}|}%
	{\gnumericPB{\raggedright}\gnumbox[l]{\textbf{4mA}}}
\\
\hhline{|-|-}
	 \multicolumn{1}{|p{\gnumericColA}|}%
	{\gnumericPB{\raggedright}\gnumbox[l]{\textbf{$r_{e1}$}}}
	&\multicolumn{1}{p{\gnumericColB}|}%
	{\gnumericPB{\raggedright}\gnumbox[l]{\textbf{$41.7\Omega$}}}
\\
\hhline{|-|-}
	 \multicolumn{1}{|p{\gnumericColA}|}%
	{\gnumericPB{\raggedright}\gnumbox[l]{\textbf{$r_{\pi2}$}}}
	&\multicolumn{1}{p{\gnumericColB}|}%
	{\gnumericPB{\raggedright}\gnumbox[l]{\textbf{2.5k\Omega}}}
\\
\hhline{|-|-}
	 \multicolumn{1}{|p{\gnumericColA}|}%
	{\gnumericPB{\raggedright}\gnumbox[l]{\textbf{$\alpha1$}}}
	&\multicolumn{1}{p{\gnumericColB}|}%
	{\gnumericPB{\raggedright}\gnumbox[l]{\textbf{0.99}}}
\\
\hhline{|-|-}
	 \multicolumn{1}{|p{\gnumericColA}|}%
	{\gnumericPB{\raggedright}\gnumbox[l]{\textbf{$g_{m2}$}}}
	&\multicolumn{1}{p{\gnumericColB}|}%
	{\gnumericPB{\raggedright}\gnumbox[l]{\textbf{40mA/V}}}
\\
\hhline{|-|-}
	 \multicolumn{1}{|p{\gnumericColA}|}%
	{\gnumericPB{\raggedright}\gnumbox[l]{\textbf{$r_{e3}$}}}
	&\multicolumn{1}{p{\gnumericColB}|}%
	{\gnumericPB{\raggedright}\gnumbox[l]{\textbf{$6.25\Omega$}}}
\\
\hhline{|-|-}
	 \multicolumn{1}{|p{\gnumericColA}|}%
	{\gnumericPB{\raggedright}\gnumbox[l]{\textbf{$r_{o3}$}}}
	&\multicolumn{1}{p{\gnumericColB}|}%
	{\gnumericPB{\raggedright}\gnumbox[l]{\textbf{$25k\Omega$}}}
\\
\hhline{|-|-}
	 \multicolumn{1}{|p{\gnumericColA}|}%
	{\gnumericPB{\raggedright}\gnumbox[l]{\textbf{$r_{\pi3}$}}}
	&\multicolumn{1}{p{\gnumericColB}|}%
	{\gnumericPB{\raggedright}\gnumbox[l]{\textbf{$625\Omega$}}}
\\



\hhline{|-|-|}
\end{tabular}

\ifthenelse{\isundefined{\languageshorthands}}{}{\languageshorthands{\languagename}}
\gnumericTableEnd


\caption{parameters}
\label{table:ee18btech11007}
\end{table}
\begin{figure}[!ht]
	\begin{center}
		
		\resizebox{\columnwidth}{!}{\begin{circuitikz}[american ]
\draw (0,0) to [R,l_=$R_{E1}$](0,-2) to node[ground]{}++(0,-0.25) ++(6,0)
(0,0) to [R,l_=$R_F$](3,0) to [R,l_=$R_{E2}$]++(0,-2)to node[ground]{}++(0,0)
(3,0)--(5,0)to [current source,l_=$-I_o$]++(0,-2) to node[ground]{}++(0,-0.5)
(0,0)--(-2,0) to [open, v^>=${V}_f$,*-] ++(0,-1) ++(6,0)
(0,0) 
;\end{circuitikz}
}
	\end{center}
\caption{circuit2}
\label{fig:circuit2}
\end{figure}
\item use feedback analysis to find open loop gain G 
\\
\solution employing loading rules in fig.\ref{fig:circuit1},we obtain circuit3 given in fig.\ref{fig:circuit3}
 \begin{figure}[!ht]
	\begin{center}
		
		\resizebox{\columnwidth}{!}{\begin{circuitikz}[american]
\draw (0,0)--(-1,0)to [voltage source,l_=$V_i$]++(0,-2)to node[ground]{}++(0,-0.5) ++(6,0)
(0.75,0) node[npn](npn1){Q1}
(npn1.C)--++(0,1.5) to [R,l_=$R_{C1}$]++(0,1.5) to node[ground,rotate=180]{}++(0,0.25)
(npn1.E)-- ++(0,-2) to [R,l_=$R_{E1}$]++(0,-1.5)to node[ground]{}++(0,-0.25)
(npn1.E)++(0,-1)to [R,l_=$R_F$]++(3,0)to[R,l_=$R_{E2}$]++(0,-1.5)to node[ground]{} ++(0,-0.25)
 (4.75,1.5)node[npn](npn2){Q2}
(npn1.C)++(0,0.75)--(npn2.B)
(npn2.E)to node[ground]{}++(0,0)
(npn2.C)--++(0,1.5)to[R,l_=$R_{C2}$]++(0,1.5)to node[ground,rotate=180]{}++(0,0.5)
(8.75,3) node[npn](npn3){Q3}
(npn2.C)++(0,0.75)--(npn3.B)
(npn3.C)to[R,l_=$R_{C3}$]++(0,1.5)to node[ground,rotate=180]{}++(0,0.25)
(npn3.E)to[short,i_=$I_o$]++(0,-2)coordinate(a)to[R,l_=$R_{E2}$]++(0,-1.5)to node[ground]{}++(0,-0.25)
(a)++(0,0.25)to[R,l_=$R_F$]++(3,0)to[R,l_=$R_{E1}$]++(0,-1.5)to node[ground]{}++(0,-0.25)

 

;\end{circuitikz}
}
	\end{center}
\caption{circuit3}
\label{fig:circuit3}
\end{figure}
\begin{figure}[!ht]
	\begin{center}
		
		\resizebox{\columnwidth}{!}{\begin{circuitikz}{american}
\draw (0,0)node[npn](npn1){Q3} ++(5,0)
(npn1.B)to[R,l_=$R_C_2$]++(-2,0) -- ++(0,-1)to node[ground]++(0,-0.25)
(npn1.E)to[R,l_=$R_o_f$]++(0,-1.5)to node[ground]++(0,-0.25)
;\end{circuitikz}}
	\end{center}
\caption{circuit4}
\label{fig:circuit4}
\end{figure}
to find $G=\frac{I_0}{V_i}$ we determine the gain of first stage,this is written by inspection as-
\begin{align}
    \frac{V_{c1}}{V_i}=\frac{-\alpha(R_{c1}||r_{\pi2})}{r_{e1}+(R_{E1}||(R_F+R_{E2}))}
\end{align}
using values from \ref{table:ee18btech11007}
\begin{align}
\frac{V_{c1}}{V_i}=-14.92V/V     
\end{align}
Next, we determine the gain of the second stage,which can be written by inspection(noting that $V_{b2}=V_{c1}$)as
\begin{align}
    \frac{V_{c2}}{V_{c1}}=-g_{m2}{R_{c2}||(h_{fe}+1)[r_{e3}+(R_{E2}||(R_F+R_{E1}))]}
\end{align}
substituting ,results in 
\begin{align}
    \frac{V_{c2}}{V_{c1}}=-131.2 V/V
\end{align}
Finally,for the third stage we can write by inspection
\begin{align}
    \frac{I_0}{V_{c2}}=\frac{I_{e3}}{V_{b3}}=\frac{1}{r_{e3}+(R_{E2}||(R_F+R_{E1}))}
\end{align}
substituing values from \ref{table:ee18btech11007} gives
\begin{align}
    \frac{I_0}{V_{c2}}=10.6mA/V
\end{align}
combining the gains of the three stags results in
\begin{align}
G=\frac{I_0}{V_i}=-14.92\times-131.2\times10.6\times10^{-3}=20.7A/V    
\end{align}
\item Find Feedback Factor H
\\
\solution feedback factor H can be found from feedback network,The feedback network consists of resistors $R_{E1},R_F,R_{E2}$ using circuit2 in  fig.\ref{fig:circuit2} we get
\begin{align}
    H=\frac{V_f}{I_0}=\frac{R_{E2}}{R_{E2}+R_F+R_{E1}} \times R_{E1}
\end{align}
\begin{align}
   H =\frac{100}{100+640+100}\times 100=11.9\Omega
\end{align}
\item Find closed loop gain T and Voltage Gain $V_0/V_s$
\\ \solution
 \begin{align}
 \label{eq:EE18BTECH11007}
    T=\frac{I_0}{V_s}=\frac{G}{1+GH}=\frac{20.7}{1+20.7\times11.9}=83.7mA/V
\end{align}
 the voltage gain is found from 
\begin{align}
    \frac{V_0}{V_s}=\frac{-I_cR_{c3}}{V_s}\approx\frac{-I_0R_{C3}}{V_s}=-TR_{C3}
    \
\end{align}
\begin{align}
    =-83.7\times10^{-3}\times600=-50.2V/V
\end{align}
\item Now assume Loop gain is large and find approximate expression for closed loop gain $T=\frac{I_o}{V_s}$
\\
\solution When GH $>>$1,
\begin{align}
    T =\frac{I_0}{V_s}\approx \frac{1}{H}
\end{align}
as
\begin{align}
    H=\frac{V_f}{I_0}=\frac{R_{E2}}{R_{E2}+R_F+R_{E1}} \times R_{E1}
\end{align}
\begin{align}
    =\frac{100}{100+640+100}\times 100=11.9\Omega
\end{align}
thus,
\begin{align}
   T =\frac{1}{11.9}=84mA/V
\end{align}
\begin{align}
 \frac{I_c}{V_s}\approx\frac{I_0}{V_s}=84 mA/V
 \end{align}
 which  we note is very close to the approximate value found in \eqref{eq:EE18BTECH11007} 
\item Find Voltage gain $\frac{V_0}{V_s}$ for above approximation
\\
\solution 
\begin{align}
\frac{V_0}{V_s}=\frac{-I_c R_{C3}}{V_s}=-84\times0.6=-50.4V/V
\end{align}
\item Find $R_{in}$ and $R_{out}$ for circuit in  fig.\ref{fig:circuit1}
\\
\solution
\begin{align}
    R_{in} =R_{if}=R_i(1+GH)
\end{align}
where $R_i$ is the input resistance of the G circuit.The value of $R_i$ can be found from the circuit in fig.\ref{fig:circuit3} as follows:
\begin{align}
    R_i=(h_{fe}+1)(r_{e1}+(R_{E1}||(R_F+R_{E2})))=13.65K\Omega
\end{align} 
\begin{align}
    R_{if}=13.65(1+20.7\times11.9)=3.38M\Omega
\end{align}
\begin{align}
    R_{of}=R_o(1+GH)
\end{align}
where $R_o$ can be determined to be 
 \begin{align}
    R_o=(R_{E2}||(R_F+R_{E1}))+r_{e3}+\frac{R_{C2}}{h_{fe}+1}
\end{align}
from values in Table \ref{table:ee18btech11007}, yields $R_o = 143.9 \Omega$. The output resistance $R_{of}$ of the feedback amplifier can now be found as
\begin{align}
    R_{of}=R_o(1+GH)=143.9(1+20.7\times11.9)=35.6K\Omega
\end{align}
$R_{out}$ is found by using circuit4 in fig.\ref{fig:circuit3}
\begin{align}
    R_{out}=r_{o3}+[R_{of}||(r_{\pi3}+R_{C2})](1+g_{m3}r_{o3}\frac{r_{\pi3}}{r_{\pi3}+R_{C2}})
\end{align}
\begin{align}
=25+[35.6||(5.625)][1+160\times25\frac{0.625}{5.625}]=2.19M\Omega \end{align}

thus $R_{out}$ is increased (from $r_{o3}$) but not by (1+GH)
\item put the obtained parameters in a table
\\
\solution 
\begin{table}[!ht]
\centering
%%%%%%%%%%%%%%%%%%%%%%%%%%%%%%%%%%%%%%%%%%%%%%%%%%%%%%%%%%%%%%%%%%%%%%
%%                                                                  %%
%%  This is the header of a LaTeX2e file exported from Gnumeric.    %%
%%                                                                  %%
%%  This file can be compiled as it stands or included in another   %%
%%  LaTeX document. The table is based on the longtable package so  %%
%%  the longtable options (headers, footers...) can be set in the   %%
%%  preamble section below (see PRAMBLE).                           %%
%%                                                                  %%
%%  To include the file in another, the following two lines must be %%
%%  in the including file:                                          %%
%%        \def\inputGnumericTable{}                                 %%
%%  at the beginning of the file and:                               %%
%%        \input{name-of-this-file.tex}                             %%
%%  where the table is to be placed. Note also that the including   %%
%%  file must use the following packages for the table to be        %%
%%  rendered correctly:                                             %%
%%    \usepackage[latin1]{inputenc}                                 %%
%%    \usepackage{color}                                            %%
%%    \usepackage{array}                                            %%
%%    \usepackage{longtable}                                        %%
%%    \usepackage{calc}                                             %%
%%    \usepackage{multirow}                                         %%
%%    \usepackage{hhline}                                           %%
%%    \usepackage{ifthen}                                           %%
%%  optionally (for landscape tables embedded in another document): %%
%%    \usepackage{lscape}                                           %%
%%                                                                  %%
%%%%%%%%%%%%%%%%%%%%%%%%%%%%%%%%%%%%%%%%%%%%%%%%%%%%%%%%%%%%%%%%%%%%%%

 

%%  This section checks if we are begin input into another file or  %%
%%  the file will be compiled alone. First use a macro taken from   %%
%%  the TeXbook ex 7.7 (suggestion of Han-Wen Nienhuys).            %%
\def\ifundefined#1{\expandafter\ifx\csname#1\endcsname\relax}


%%  Check for the \def token for inputed files. If it is not        %%
%%  defined, the file will be processed as a standalone and the     %%
%%  preamble will be used.                                          %%
\ifundefined{inputGnumericTable}

%%  We must be able to close or not the document at the end.        %%
	\def\gnumericTableEnd{\end{document}}


%%%%%%%%%%%%%%%%%%%%%%%%%%%%%%%%%%%%%%%%%%%%%%%%%%%%%%%%%%%%%%%%%%%%%%
%%                                                                  %%
%%  This is the PREAMBLE. Change these values to get the right      %%
%%  paper size and other niceties.                                  %%
%%                                                                  %%
%%%%%%%%%%%%%%%%%%%%%%%%%%%%%%%%%%%%%%%%%%%%%%%%%%%%%%%%%%%%%%%%%%%%%%

	\documentclass[12pt%
			  %,landscape%
                    ]{report}
       \usepackage[latin1]{inputenc}
       \usepackage{fullpage}
       \usepackage{color}
       \usepackage{array}
       \usepackage{longtable}
       \usepackage{calc}
       \usepackage{multirow}
       \usepackage{hhline}
       \usepackage{ifthen}

	


%%  End of the preamble for the standalone. The next section is for %%
%%  documents which are included into other LaTeX2e files.          %%
\else

%%  We are not a stand alone document. For a regular table, we will %%
%%  have no preamble and only define the closing to mean nothing.   %%
    \def\gnumericTableEnd{}

%%  If we want landscape mode in an embedded document, comment out  %%
%%  the line above and uncomment the two below. The table will      %%
%%  begin on a new page and run in landscape mode.                  %%
%       \def\gnumericTableEnd{\end{landscape}}
%       \begin{landscape}


%%  End of the else clause for this file being \input.              %%
\fi

%%%%%%%%%%%%%%%%%%%%%%%%%%%%%%%%%%%%%%%%%%%%%%%%%%%%%%%%%%%%%%%%%%%%%%
%%                                                                  %%
%%  The rest is the gnumeric table, except for the closing          %%
%%  statement. Changes below will alter the table's appearance.     %%
%%                                                                  %%
%%%%%%%%%%%%%%%%%%%%%%%%%%%%%%%%%%%%%%%%%%%%%%%%%%%%%%%%%%%%%%%%%%%%%%

\providecommand{\gnumericmathit}[1]{#1} 
%%  Uncomment the next line if you would like your numbers to be in %%
%%  italics if they are italizised in the gnumeric table.           %%
%\renewcommand{\gnumericmathit}[1]{\mathit{#1}}
\providecommand{\gnumericPB}[1]%
{\let\gnumericTemp=\\#1\let\\=\gnumericTemp\hspace{0pt}}
 \ifundefined{gnumericTableWidthDefined}
        \newlength{\gnumericTableWidth}
        \newlength{\gnumericTableWidthComplete}
        \newlength{\gnumericMultiRowLength}
        \global\def\gnumericTableWidthDefined{}
 \fi
%% The following setting protects this code from babel shorthands.  %%
 \ifthenelse{\isundefined{\languageshorthands}}{}{\languageshorthands{english}}
%%  The default table format retains the relative column widths of  %%
%%  gnumeric. They can easily be changed to c, r or l. In that case %%
%%  you may want to comment out the next line and uncomment the one %%
%%  thereafter                                                      %%
\providecommand\gnumbox{\makebox[0pt]}
%%\providecommand\gnumbox[1][]{\makebox}

%% to adjust positions in multirow situations                       %%
\setlength{\bigstrutjot}{\jot}
\setlength{\extrarowheight}{\doublerulesep}

%%  The \setlongtables command keeps column widths the same across  %%
%%  pages. Simply comment out next line for varying column widths.  %%
\setlongtables

\setlength\gnumericTableWidth{%
	83pt+%
	91pt+%
0pt}
\def\gumericNumCols{2}
\setlength\gnumericTableWidthComplete{\gnumericTableWidth+%
         \tabcolsep*\gumericNumCols*2+\arrayrulewidth*\gumericNumCols}
\ifthenelse{\lengthtest{\gnumericTableWidthComplete > \linewidth}}%
         {\def\gnumericScale{\ratio{\linewidth-%
                        \tabcolsep*\gumericNumCols*2-%
                        \arrayrulewidth*\gumericNumCols}%
{\gnumericTableWidth}}}%
{\def\gnumericScale{1}}

%%%%%%%%%%%%%%%%%%%%%%%%%%%%%%%%%%%%%%%%%%%%%%%%%%%%%%%%%%%%%%%%%%%%%%
%%                                                                  %%
%% The following are the widths of the various columns. We are      %%
%% defining them here because then they are easier to change.       %%
%% Depending on the cell formats we may use them more than once.    %%
%%                                                                  %%
%%%%%%%%%%%%%%%%%%%%%%%%%%%%%%%%%%%%%%%%%%%%%%%%%%%%%%%%%%%%%%%%%%%%%%

\ifthenelse{\isundefined{\gnumericColA}}{\newlength{\gnumericColA}}{}\settowidth{\gnumericColA}{\begin{tabular}{@{}p{83pt*\gnumericScale}@{}}x\end{tabular}}
\ifthenelse{\isundefined{\gnumericColB}}{\newlength{\gnumericColB}}{}\settowidth{\gnumericColB}{\begin{tabular}{@{}p{91pt*\gnumericScale}@{}}x\end{tabular}}

\begin{tabular}[c]{%
	b{\gnumericColA}%
	b{\gnumericColB}%
	}

%%%%%%%%%%%%%%%%%%%%%%%%%%%%%%%%%%%%%%%%%%%%%%%%%%%%%%%%%%%%%%%%%%%%%%
%%  The longtable options. (Caption, headers... see Goosens, p.124) %%
%	\caption{The Table Caption.}             \\	%
% \hline	% Across the top of the table.
%%  The rest of these options are table rows which are placed on    %%
%%  the first, last or every page. Use \multicolumn if you want.    %%

%%  Header for the first page.                                      %%
%	\multicolumn{2}{c}{The First Header} \\ \hline 
%	\multicolumn{1}{c}{colTag}	%Column 1
%	&\multicolumn{1}{c}{colTag}	\\ \hline %Last column
%	\endfirsthead

%%  The running header definition.                                  %%
%	\hline
%	\multicolumn{2}{l}{\ldots\small\slshape continued} \\ \hline
%	\multicolumn{1}{c}{colTag}	%Column 1
%	&\multicolumn{1}{c}{colTag}	\\ \hline %Last column
%	\endhead

%%  The running footer definition.                                  %%
%	\hline
%	\multicolumn{2}{r}{\small\slshape continued\ldots} \\
%	\endfoot

%%  The ending footer definition.                                   %%
%	\multicolumn{2}{c}{That's all folks} \\ \hline 
%	\endlastfoot
%%%%%%%%%%%%%%%%%%%%%%%%%%%%%%%%%%%%%%%%%%%%%%%%%%%%%%%%%%%%%%%%%%%%%%

\hhline{|-|-}
	 \multicolumn{1}{|p{\gnumericColA}|}%
	{\gnumericPB{\raggedright}\gnumbox[l]{\textbf{Parameter}}}
	&\multicolumn{1}{p{\gnumericColA}|}%
	{\gnumericPB{\raggedright}\gnumbox[l]{\textbf{Value}}}
\\
\hhline{|-|-}
	 \multicolumn{1}{|p{\gnumericColA}|}%
	{\gnumericPB{\raggedright}\gnumbox[l]{\textbf{G}}}
	&\multicolumn{1}{p{\gnumericColA}|}%
	{\gnumericPB{\raggedright}\gnumbox[l]{\textbf{20.7A/V}}}
\\
\hhline{|-|-}
	 \multicolumn{1}{|p{\gnumericColA}|}%
	{\gnumericPB{\raggedright}\gnumbox[l]{\textbf{H}}}
	&\multicolumn{1}{p{\gnumericColA}|}%
	{\gnumericPB{\raggedright}\gnumbox[l]{\textbf{$11.9\Omega$}}}
\\
\hhline{|-|-}
	 \multicolumn{1}{|p{\gnumericColA}|}%
	{\gnumericPB{\raggedright}\gnumbox[l]{\textbf{T}}}
	&\multicolumn{1}{p{\gnumericColA}|}%
	{\gnumericPB{\raggedright}\gnumbox[l]{\textbf{83.7mA/V}}}
\\
\hhline{|-|-}
	 \multicolumn{1}{|p{\gnumericColA}|}%
	{\gnumericPB{\raggedright}\gnumbox[l]{\textbf{$V_o$/$V_s$}}}
	&\multicolumn{1}{p{\gnumericColA}|}%
	{\gnumericPB{\raggedright}\gnumbox[l]{\textbf{-50.2V/V}}}
\\
\hhline{|-|-}
	 \multicolumn{1}{|p{\gnumericColA}|}%
	{\gnumericPB{\raggedright}\gnumbox[l]{\textbf{$R_{in}$}}}
	&\multicolumn{1}{p{\gnumericColB}|}%
	{\gnumericPB{\raggedright}\gnumbox[l]{\textbf{$3.38M\Omega$}}}
\\
\hhline{|-|-}
	 \multicolumn{1}{|p{\gnumericColA}|}%
	{\gnumericPB{\raggedright}\gnumbox[l]{\textbf{$R_{out}$}}}
	&\multicolumn{1}{p{\gnumericColA}|}%
	{\gnumericPB{\raggedright}\gnumbox[l]{\textbf{$2.19M\Omega$}}}
\\
\hhline{|-|-}
	 \multicolumn{1}{|p{\gnumericColA}|}%
	{\gnumericPB{\raggedright}\gnumbox[l]{\textbf{$R_{of}$}}}
	&\multicolumn{1}{p{\gnumericColA}|}%
	{\gnumericPB{\raggedright}\gnumbox[l]{\textbf{$35.6k\Omega$}}}
\\



\hhline{|-|-|}
\end{tabular}

\ifthenelse{\isundefined{\languageshorthands}}{}{\languageshorthands{\languagename}}
\gnumericTableEnd

\caption{parameters}
\label{table:parameters}
\end{table}
\item Represent this amplifier in  a control system Block Diagram
\\
\solution figure in  fig.\ref{fig:block_diagram} represents our control system
\begin{figure}[!ht]
	\begin{center}
		
		\resizebox{\columnwidth}{!}{\tikzstyle{block} = [draw, rectangle, 
    minimum height=1.25em, minimum width=2.5em]
\tikzstyle{sum} = [draw, circle, node distance=1cm]
\tikzstyle{input} = [coordinate]
\tikzstyle{output} = [coordinate]
\tikzstyle{pinstyle} = [pin edge={to-,thin,black}]


\begin{tikzpicture}[auto, node distance=2.5cm,>=latex']
   
    \node [input, name=input] {};
    \node [sum, right of=input] (sum) {};
    \node [block, right of=sum] (controller) {$G$};
    
    \node [output, right of=controller] (output) {};
    \node [block, below of=controller] (measurements) {$H$};

   \draw [draw,->] (input) -- node[pos=0.99] {$+$} node {$V_{s}$} (sum);
    \draw [->] (sum) -- node {$V_{i}$} (controller);
    \draw [->] (controller) -- node [name=y] {$I_{o}$}(output);
    \draw [->] (y) |- (measurements);
    \draw [->] (measurements) -| node[pos=0.99] {$-$} node [near end] {$V_{f}$} (sum);
\end{tikzpicture}}
	\end{center}
\caption{block diagram}
\label{fig:block_diagram}
\end{figure}
\item write a code for doing calculations and verify the values obtained in \ref{table:parameters} 
\\
\solution 
following code does all the calculations of above equations to give parameters in
\ref{table:parameters} 
\begin{lstlisting}
codes/ee18btech11007/circuit_calc.py
\end{lstlisting}
}%



\end{enumerate}
