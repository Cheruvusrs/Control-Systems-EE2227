\begin{enumerate}[label=\thesection.\arabic*.,ref=\thesection.\theenumi]
\numberwithin{equation}{enumi}

\item The Block diagram of a system is illustrated in the figure shown, where $X(s)$ is the input and $Y(s)$ is the output.  Draw the equivalent signal flow graph. 
\renewcommand{\thefigure}{\theenumi.\arabic{figure}}
%
\begin{figure}[!ht]
    \begin{center}
		
		\resizebox{\columnwidth}{!}{\tikzstyle{block} = [draw, rectangle, 
    minimum height=1.25em, minimum width=2.5em]
\tikzstyle{sum} = [draw, circle, node distance=1cm]
\tikzstyle{input} = [coordinate]
\tikzstyle{output} = [coordinate]
\tikzstyle{pinstyle} = [pin edge={to-,thin,black}]


\begin{tikzpicture}[auto, node distance=2.5cm,>=latex']
   
    \node [input, name=input] {};
    \node [sum, right of=input] (sum) {};
    \node [block, right of=sum] (controller) {$G$};
    
    \node [output, right of=controller] (output) {};
    \node [block, below of=controller] (measurements) {$H$};

   \draw [draw,->] (input) -- node[pos=0.99] {$+$} node {$V_{s}$} (sum);
    \draw [->] (sum) -- node {$V_{i}$} (controller);
    \draw [->] (controller) -- node [name=y] {$I_{o}$}(output);
    \draw [->] (y) |- (measurements);
    \draw [->] (measurements) -| node[pos=0.99] {$-$} node [near end] {$V_{f}$} (sum);
\end{tikzpicture}}
	\end{center}
\caption{Block Diagram}
\label{fig:ee18btech11003_block_diagram}
\end{figure}
\\
\solution The signal flow graph of the block diagram in Fig. \ref{fig:ee18btech11003_block_diagram} is available in Fig. \ref{fig:ee18btech11003_signal_flow}
%
\begin{figure}[!ht]
\begin{center}
		
		\resizebox{\columnwidth}{!}{\input{./figs/ee18btech11003/signal_flow.tex}}
	\end{center}
\caption{Signal Flow Graph}
\label{fig:ee18btech11003_signal_flow}
\end{figure}
%
\renewcommand{\thefigure}{\theenumi}
\item Draw all the forward paths in Fig. \ref{fig:ee18btech11003_signal_flow}
and compute the respective gains.
\renewcommand{\thefigure}{\theenumi.\arabic{figure}}
\\
\solution The forward paths are available in Figs. \ref{fig:ee18btech11003_P1}
 and \ref{fig:ee18btech11003_P2}.  The respective gains are
\begin{align}
P_1&=s \brak{\frac{1}{s}}=1
\\
P_2&=(1/s)(1/s)=1/s^2
\end{align}
%
\begin{figure}[!ht]
\begin{center}
		
		\resizebox{\columnwidth}{!}{\input{./figs/ee18btech11003/P1.tex}}
	\end{center}
\caption{$P_1$}
\label{fig:ee18btech11003_P1}
\end{figure}
%
\begin{figure}[!ht]
\begin{center}
		
		\resizebox{\columnwidth}{!}{\input{./figs/ee18btech11003/P2.tex}}
	\end{center}
\caption{$P_2$}
\label{fig:ee18btech11003_P2}
\end{figure}
\renewcommand{\thefigure}{\theenumi}
%
\item Draw all the loops in Fig. \ref{fig:ee18btech11003_signal_flow} and calculate the respective gains.
\renewcommand{\thefigure}{\theenumi.\arabic{figure}}
\\
\solution The loops are available in Figs. \ref{fig:ee18btech11003_L1}-\ref{fig:ee18btech11003_L4}
and the corresponding gains are
%
\begin{align}
L_1&=(-1)(s)=-s
\\
L_2&=s\brak{\frac{1}{s}}\brak{-1}=-1
\\
L_3&=\brak{\frac{1}{s}}(-1)=-\frac{1}{s}
\\
L_4&=\brak{\frac{1}{s}}\brak{\frac{1}{s}}(-1)=-\frac{1}{s^2}
\end{align}

\begin{figure}[!ht]
\begin{center}
		
		\resizebox{\columnwidth}{!}{\input{./figs/ee18btech11003/L1.tex}}
	\end{center}
\caption{$L_1$}
\label{fig:ee18btech11003_L1}
\end{figure}



\begin{figure}[!ht]
\begin{center}
		
		\resizebox{\columnwidth}{!}{\input{./figs/ee18btech11003/L2.tex}}
	\end{center}
\caption{$L_2$}
\label{fig:ee18btech11003_L2}
\end{figure}



\begin{figure}[!ht]
\begin{center}
		
		\resizebox{\columnwidth}{!}{\input{./figs/ee18btech11003/L3.tex}}
	\end{center}
\caption{$L_3$}
\label{fig:ee18btech11003_L3}
\end{figure}



\begin{figure}[!ht]
\begin{center}
		
		\resizebox{\columnwidth}{!}{\input{./figs/ee18btech11003/L4.tex}}
	\end{center}
\caption{$L_4$}
\label{fig:ee18btech11003_L4}
\end{figure}

\renewcommand{\thefigure}{\theenumi}

\item State Mason's Gain formula and explain the parameters through a table.
\\
\solution 
According to Mason's Gain Formula,
\begin{align}
T &= \frac{Y(s)}{X(s)} 
\\
 &= \frac{\sum_{i=1}^{N} P_i\Delta_i}{\Delta}
\label{eq:ee18btech11003_mason}
\end{align}
%
where the parameters are described in Table \ref{table:ee18btech11003}
\begin{table}[!ht]
\centering
\begin{enumerate}[label=\thesection.\arabic*.,ref=\thesection.\theenumi]
\numberwithin{equation}{enumi}

\item The Block diagram of a system is illustrated in the figure shown, where $X(s)$ is the input and $Y(s)$ is the output.  Draw the equivalent signal flow graph. 
\renewcommand{\thefigure}{\theenumi.\arabic{figure}}
%
\begin{figure}[!ht]
    \begin{center}
		
		\resizebox{\columnwidth}{!}{\tikzstyle{block} = [draw, rectangle, 
    minimum height=1.25em, minimum width=2.5em]
\tikzstyle{sum} = [draw, circle, node distance=1cm]
\tikzstyle{input} = [coordinate]
\tikzstyle{output} = [coordinate]
\tikzstyle{pinstyle} = [pin edge={to-,thin,black}]


\begin{tikzpicture}[auto, node distance=2.5cm,>=latex']
   
    \node [input, name=input] {};
    \node [sum, right of=input] (sum) {};
    \node [block, right of=sum] (controller) {$G$};
    
    \node [output, right of=controller] (output) {};
    \node [block, below of=controller] (measurements) {$H$};

   \draw [draw,->] (input) -- node[pos=0.99] {$+$} node {$V_{s}$} (sum);
    \draw [->] (sum) -- node {$V_{i}$} (controller);
    \draw [->] (controller) -- node [name=y] {$I_{o}$}(output);
    \draw [->] (y) |- (measurements);
    \draw [->] (measurements) -| node[pos=0.99] {$-$} node [near end] {$V_{f}$} (sum);
\end{tikzpicture}}
	\end{center}
\caption{Block Diagram}
\label{fig:ee18btech11003_block_diagram}
\end{figure}
\\
\solution The signal flow graph of the block diagram in Fig. \ref{fig:ee18btech11003_block_diagram} is available in Fig. \ref{fig:ee18btech11003_signal_flow}
%
\begin{figure}[!ht]
\begin{center}
		
		\resizebox{\columnwidth}{!}{\input{./figs/ee18btech11003/signal_flow.tex}}
	\end{center}
\caption{Signal Flow Graph}
\label{fig:ee18btech11003_signal_flow}
\end{figure}
%
\renewcommand{\thefigure}{\theenumi}
\item Draw all the forward paths in Fig. \ref{fig:ee18btech11003_signal_flow}
and compute the respective gains.
\renewcommand{\thefigure}{\theenumi.\arabic{figure}}
\\
\solution The forward paths are available in Figs. \ref{fig:ee18btech11003_P1}
 and \ref{fig:ee18btech11003_P2}.  The respective gains are
\begin{align}
P_1&=s \brak{\frac{1}{s}}=1
\\
P_2&=(1/s)(1/s)=1/s^2
\end{align}
%
\begin{figure}[!ht]
\begin{center}
		
		\resizebox{\columnwidth}{!}{\input{./figs/ee18btech11003/P1.tex}}
	\end{center}
\caption{$P_1$}
\label{fig:ee18btech11003_P1}
\end{figure}
%
\begin{figure}[!ht]
\begin{center}
		
		\resizebox{\columnwidth}{!}{\input{./figs/ee18btech11003/P2.tex}}
	\end{center}
\caption{$P_2$}
\label{fig:ee18btech11003_P2}
\end{figure}
\renewcommand{\thefigure}{\theenumi}
%
\item Draw all the loops in Fig. \ref{fig:ee18btech11003_signal_flow} and calculate the respective gains.
\renewcommand{\thefigure}{\theenumi.\arabic{figure}}
\\
\solution The loops are available in Figs. \ref{fig:ee18btech11003_L1}-\ref{fig:ee18btech11003_L4}
and the corresponding gains are
%
\begin{align}
L_1&=(-1)(s)=-s
\\
L_2&=s\brak{\frac{1}{s}}\brak{-1}=-1
\\
L_3&=\brak{\frac{1}{s}}(-1)=-\frac{1}{s}
\\
L_4&=\brak{\frac{1}{s}}\brak{\frac{1}{s}}(-1)=-\frac{1}{s^2}
\end{align}

\begin{figure}[!ht]
\begin{center}
		
		\resizebox{\columnwidth}{!}{\input{./figs/ee18btech11003/L1.tex}}
	\end{center}
\caption{$L_1$}
\label{fig:ee18btech11003_L1}
\end{figure}



\begin{figure}[!ht]
\begin{center}
		
		\resizebox{\columnwidth}{!}{\input{./figs/ee18btech11003/L2.tex}}
	\end{center}
\caption{$L_2$}
\label{fig:ee18btech11003_L2}
\end{figure}



\begin{figure}[!ht]
\begin{center}
		
		\resizebox{\columnwidth}{!}{\input{./figs/ee18btech11003/L3.tex}}
	\end{center}
\caption{$L_3$}
\label{fig:ee18btech11003_L3}
\end{figure}



\begin{figure}[!ht]
\begin{center}
		
		\resizebox{\columnwidth}{!}{\input{./figs/ee18btech11003/L4.tex}}
	\end{center}
\caption{$L_4$}
\label{fig:ee18btech11003_L4}
\end{figure}

\renewcommand{\thefigure}{\theenumi}

\item State Mason's Gain formula and explain the parameters through a table.
\\
\solution 
According to Mason's Gain Formula,
\begin{align}
T &= \frac{Y(s)}{X(s)} 
\\
 &= \frac{\sum_{i=1}^{N} P_i\Delta_i}{\Delta}
\label{eq:ee18btech11003_mason}
\end{align}
%
where the parameters are described in Table \ref{table:ee18btech11003}
\begin{table}[!ht]
\centering
\begin{enumerate}[label=\thesection.\arabic*.,ref=\thesection.\theenumi]
\numberwithin{equation}{enumi}

\item The Block diagram of a system is illustrated in the figure shown, where $X(s)$ is the input and $Y(s)$ is the output.  Draw the equivalent signal flow graph. 
\renewcommand{\thefigure}{\theenumi.\arabic{figure}}
%
\begin{figure}[!ht]
    \begin{center}
		
		\resizebox{\columnwidth}{!}{\tikzstyle{block} = [draw, rectangle, 
    minimum height=1.25em, minimum width=2.5em]
\tikzstyle{sum} = [draw, circle, node distance=1cm]
\tikzstyle{input} = [coordinate]
\tikzstyle{output} = [coordinate]
\tikzstyle{pinstyle} = [pin edge={to-,thin,black}]


\begin{tikzpicture}[auto, node distance=2.5cm,>=latex']
   
    \node [input, name=input] {};
    \node [sum, right of=input] (sum) {};
    \node [block, right of=sum] (controller) {$G$};
    
    \node [output, right of=controller] (output) {};
    \node [block, below of=controller] (measurements) {$H$};

   \draw [draw,->] (input) -- node[pos=0.99] {$+$} node {$V_{s}$} (sum);
    \draw [->] (sum) -- node {$V_{i}$} (controller);
    \draw [->] (controller) -- node [name=y] {$I_{o}$}(output);
    \draw [->] (y) |- (measurements);
    \draw [->] (measurements) -| node[pos=0.99] {$-$} node [near end] {$V_{f}$} (sum);
\end{tikzpicture}}
	\end{center}
\caption{Block Diagram}
\label{fig:ee18btech11003_block_diagram}
\end{figure}
\\
\solution The signal flow graph of the block diagram in Fig. \ref{fig:ee18btech11003_block_diagram} is available in Fig. \ref{fig:ee18btech11003_signal_flow}
%
\begin{figure}[!ht]
\begin{center}
		
		\resizebox{\columnwidth}{!}{\input{./figs/ee18btech11003/signal_flow.tex}}
	\end{center}
\caption{Signal Flow Graph}
\label{fig:ee18btech11003_signal_flow}
\end{figure}
%
\renewcommand{\thefigure}{\theenumi}
\item Draw all the forward paths in Fig. \ref{fig:ee18btech11003_signal_flow}
and compute the respective gains.
\renewcommand{\thefigure}{\theenumi.\arabic{figure}}
\\
\solution The forward paths are available in Figs. \ref{fig:ee18btech11003_P1}
 and \ref{fig:ee18btech11003_P2}.  The respective gains are
\begin{align}
P_1&=s \brak{\frac{1}{s}}=1
\\
P_2&=(1/s)(1/s)=1/s^2
\end{align}
%
\begin{figure}[!ht]
\begin{center}
		
		\resizebox{\columnwidth}{!}{\input{./figs/ee18btech11003/P1.tex}}
	\end{center}
\caption{$P_1$}
\label{fig:ee18btech11003_P1}
\end{figure}
%
\begin{figure}[!ht]
\begin{center}
		
		\resizebox{\columnwidth}{!}{\input{./figs/ee18btech11003/P2.tex}}
	\end{center}
\caption{$P_2$}
\label{fig:ee18btech11003_P2}
\end{figure}
\renewcommand{\thefigure}{\theenumi}
%
\item Draw all the loops in Fig. \ref{fig:ee18btech11003_signal_flow} and calculate the respective gains.
\renewcommand{\thefigure}{\theenumi.\arabic{figure}}
\\
\solution The loops are available in Figs. \ref{fig:ee18btech11003_L1}-\ref{fig:ee18btech11003_L4}
and the corresponding gains are
%
\begin{align}
L_1&=(-1)(s)=-s
\\
L_2&=s\brak{\frac{1}{s}}\brak{-1}=-1
\\
L_3&=\brak{\frac{1}{s}}(-1)=-\frac{1}{s}
\\
L_4&=\brak{\frac{1}{s}}\brak{\frac{1}{s}}(-1)=-\frac{1}{s^2}
\end{align}

\begin{figure}[!ht]
\begin{center}
		
		\resizebox{\columnwidth}{!}{\input{./figs/ee18btech11003/L1.tex}}
	\end{center}
\caption{$L_1$}
\label{fig:ee18btech11003_L1}
\end{figure}



\begin{figure}[!ht]
\begin{center}
		
		\resizebox{\columnwidth}{!}{\input{./figs/ee18btech11003/L2.tex}}
	\end{center}
\caption{$L_2$}
\label{fig:ee18btech11003_L2}
\end{figure}



\begin{figure}[!ht]
\begin{center}
		
		\resizebox{\columnwidth}{!}{\input{./figs/ee18btech11003/L3.tex}}
	\end{center}
\caption{$L_3$}
\label{fig:ee18btech11003_L3}
\end{figure}



\begin{figure}[!ht]
\begin{center}
		
		\resizebox{\columnwidth}{!}{\input{./figs/ee18btech11003/L4.tex}}
	\end{center}
\caption{$L_4$}
\label{fig:ee18btech11003_L4}
\end{figure}

\renewcommand{\thefigure}{\theenumi}

\item State Mason's Gain formula and explain the parameters through a table.
\\
\solution 
According to Mason's Gain Formula,
\begin{align}
T &= \frac{Y(s)}{X(s)} 
\\
 &= \frac{\sum_{i=1}^{N} P_i\Delta_i}{\Delta}
\label{eq:ee18btech11003_mason}
\end{align}
%
where the parameters are described in Table \ref{table:ee18btech11003}
\begin{table}[!ht]
\centering
\begin{enumerate}[label=\thesection.\arabic*.,ref=\thesection.\theenumi]
\numberwithin{equation}{enumi}

\item The Block diagram of a system is illustrated in the figure shown, where $X(s)$ is the input and $Y(s)$ is the output.  Draw the equivalent signal flow graph. 
\renewcommand{\thefigure}{\theenumi.\arabic{figure}}
%
\begin{figure}[!ht]
    \begin{center}
		
		\resizebox{\columnwidth}{!}{\input{./figs/ee18btech11003/block_diagram.tex}}
	\end{center}
\caption{Block Diagram}
\label{fig:ee18btech11003_block_diagram}
\end{figure}
\\
\solution The signal flow graph of the block diagram in Fig. \ref{fig:ee18btech11003_block_diagram} is available in Fig. \ref{fig:ee18btech11003_signal_flow}
%
\begin{figure}[!ht]
\begin{center}
		
		\resizebox{\columnwidth}{!}{\input{./figs/ee18btech11003/signal_flow.tex}}
	\end{center}
\caption{Signal Flow Graph}
\label{fig:ee18btech11003_signal_flow}
\end{figure}
%
\renewcommand{\thefigure}{\theenumi}
\item Draw all the forward paths in Fig. \ref{fig:ee18btech11003_signal_flow}
and compute the respective gains.
\renewcommand{\thefigure}{\theenumi.\arabic{figure}}
\\
\solution The forward paths are available in Figs. \ref{fig:ee18btech11003_P1}
 and \ref{fig:ee18btech11003_P2}.  The respective gains are
\begin{align}
P_1&=s \brak{\frac{1}{s}}=1
\\
P_2&=(1/s)(1/s)=1/s^2
\end{align}
%
\begin{figure}[!ht]
\begin{center}
		
		\resizebox{\columnwidth}{!}{\input{./figs/ee18btech11003/P1.tex}}
	\end{center}
\caption{$P_1$}
\label{fig:ee18btech11003_P1}
\end{figure}
%
\begin{figure}[!ht]
\begin{center}
		
		\resizebox{\columnwidth}{!}{\input{./figs/ee18btech11003/P2.tex}}
	\end{center}
\caption{$P_2$}
\label{fig:ee18btech11003_P2}
\end{figure}
\renewcommand{\thefigure}{\theenumi}
%
\item Draw all the loops in Fig. \ref{fig:ee18btech11003_signal_flow} and calculate the respective gains.
\renewcommand{\thefigure}{\theenumi.\arabic{figure}}
\\
\solution The loops are available in Figs. \ref{fig:ee18btech11003_L1}-\ref{fig:ee18btech11003_L4}
and the corresponding gains are
%
\begin{align}
L_1&=(-1)(s)=-s
\\
L_2&=s\brak{\frac{1}{s}}\brak{-1}=-1
\\
L_3&=\brak{\frac{1}{s}}(-1)=-\frac{1}{s}
\\
L_4&=\brak{\frac{1}{s}}\brak{\frac{1}{s}}(-1)=-\frac{1}{s^2}
\end{align}

\begin{figure}[!ht]
\begin{center}
		
		\resizebox{\columnwidth}{!}{\input{./figs/ee18btech11003/L1.tex}}
	\end{center}
\caption{$L_1$}
\label{fig:ee18btech11003_L1}
\end{figure}



\begin{figure}[!ht]
\begin{center}
		
		\resizebox{\columnwidth}{!}{\input{./figs/ee18btech11003/L2.tex}}
	\end{center}
\caption{$L_2$}
\label{fig:ee18btech11003_L2}
\end{figure}



\begin{figure}[!ht]
\begin{center}
		
		\resizebox{\columnwidth}{!}{\input{./figs/ee18btech11003/L3.tex}}
	\end{center}
\caption{$L_3$}
\label{fig:ee18btech11003_L3}
\end{figure}



\begin{figure}[!ht]
\begin{center}
		
		\resizebox{\columnwidth}{!}{\input{./figs/ee18btech11003/L4.tex}}
	\end{center}
\caption{$L_4$}
\label{fig:ee18btech11003_L4}
\end{figure}

\renewcommand{\thefigure}{\theenumi}

\item State Mason's Gain formula and explain the parameters through a table.
\\
\solution 
According to Mason's Gain Formula,
\begin{align}
T &= \frac{Y(s)}{X(s)} 
\\
 &= \frac{\sum_{i=1}^{N} P_i\Delta_i}{\Delta}
\label{eq:ee18btech11003_mason}
\end{align}
%
where the parameters are described in Table \ref{table:ee18btech11003}
\begin{table}[!ht]
\centering
\input{./tables/ee18btech11003.tex}
\caption{}
\label{table:ee18btech11003}
\end{table}
\item List the parameters in Table \ref{table:ee18btech11003}
for Fig. \ref{fig:ee18btech11003_signal_flow}.
\\
\solution The parameters are available in Table \ref{table:ee18btech11003_ex}

\begin{table}[!ht]
\centering
\input{./tables/ee18btech11003_ex.tex}
\caption{}
\label{table:ee18btech11003_ex}
\end{table}

\item  Find the transfer function using Mason's Gain Formula.
\renewcommand{\thefigure}{\theenumi.\arabic{figure}}
%
\\
\solution From \eqref{eq:ee18btech11003_mason} and \ref{table:ee18btech11003_ex},
\begin{align}
T(s)&=\frac{P_1 \Delta_1+P_2 \Delta_2}{\Delta}
\\
&=\frac{1 +\frac{1}{s^2}}{1-(-s-1-\frac{1}{s}-\frac{1}{s^2})}
\\
&=\frac{s^2+1}{s^3+2s^2+s+1}
\end{align}
%
after simplification.
\renewcommand{\thefigure}{\theenumi}
\item State the Equivalent Matrix Form of  Masons Gain formula 
\\
\solution Mason's rule can be stated in a simple matrix form. Assume T is  the transient matrix of the graph where \[t_{nm}\ =[T_{nm}]\] is sum transmittance of branches from node m toward node n. Then, the gain from node m to node n of the graph is equal to \[u_{nm}\ =[U_{nm}]\] where,
\begin{align}
    U=(I-T)^-1
\end{align}
and I is the identity matrix.
\item Find the Transfer Function using the matrix equivalent form of Masons Gain Formula
\\
\solution The transient matrix for \ref{fig:ee18btech11003_signal_flow}  is 
\begin{align}
\vec{T} = \myvec{0 & 1 & 0 & 0 & 0 \ \\
0 & 0 & s+1/s & 0 & 0   \\
 0 & 0 & 0 & 1 & 0 \\
 -1 & 0 & 0 & 0 &1/s \\
-1 & 0 & 0 & 0 & 0 }
\end{align}
\begin{align}
\vec{I} = \myvec{1 & 0 & 0 & 0 & 0 \ \\
0 & 1 & 0 & 0 & 0   \\
 0 & 0 & 1 & 0 & 0 \\
 0 & 0 & 0 & 1 & 0 \\
0 & 0 & 0 & 0 & 1 }
\end{align}
\begin{align}
\vec{I}-\vec{T} = \myvec{1 & -1 & 0 & 0 & 0 \ \\
0 & 1 & -s-1/s & 0 & 0   \\
 0 & 0 & 1 & -1 & 0 \\
 1 & 0 & 0 & 1 &-1/s \\
1 & 0 & 0 & 0 & 1 }
\end{align}
\begin{align}
    U =(\vec{I}-\vec{T})^-1
\end{align}
our transfer function is $U_0_4$
\begin{align}
U_0_4=C_4_0/|\vec{I}-\vec{T}|
\label{eq:transfer_function}
\end{align}
\begin{align}
    C_4_0 &=\mydet{
-1&0&0&0 
\\1 &-s-1/s&0&0 
\\0&1&-1&0
\\0&0&1&-1/s}
\end{align}
\begin{align}
    C_4_0 &=-1\mydet{
-s-1/s&0&0\\1&-1&0 
\\0&1&-1/s}
\end{align}
\begin{align}
    C_4_0 &=(s+1/s)(1/s)=\frac{s^2+1}{s^2}
    \label{eq:C40}
\end{align}
\begin{align}
|\vec{I}-\vec{T}|=1\mydet{1&-s-1/s&0&0 \\0&1&-1&0 \\0&0&1&-1/s\\0&0&0&1}  \\ +1\mydet{0&-s-1/s&0&0 \\0&1&-1&0 \\1&0&1&-1/s\\1&0&0&1}
\end{align}
\begin{align}
|\vec{I}-\vec{T}|= 1\mydet{1&-1&0\\0&1&-1/s\\0&0&1}+(s+1/s)\mydet{0&-1&0\\0&1&-1/s\\0&0&1}\ \\+(s+1/s)\mydet{0&-1&0\\1&1&-1/s\\1&0&1}   
\end{align}
\begin{align}
|\vec{I}-\vec{T}|=1+0+(s+1/s)(1+1/s)
\end{align}
\begin{align}
|\vec{I}-\vec{T}|=s+2+ 1/s +1/s^2
\end{align}
\begin{align}
 |\vec{I}-\vec{T}|=\frac{s^3+2s^2+s+1}{s^2}
 \label{eq:det}
\end{align}
from \eqref{eq:transfer_function},\eqref{eq:C40},\eqref{eq:det} we get
\begin{align}
    T(s)=\frac{s^2+1}{s^3+2s^2+s+1}
\end{align}
\\
\item Write a program to compute Mason's gain formula, given the branch nodes and gains for each path.
\\
\solution below code finds Masons Gain 
\begin{lstlisting}
codes/MasonsGain.py
\end{lstlisting}
\end{enumerate}

\caption{}
\label{table:ee18btech11003}
\end{table}
\item List the parameters in Table \ref{table:ee18btech11003}
for Fig. \ref{fig:ee18btech11003_signal_flow}.
\\
\solution The parameters are available in Table \ref{table:ee18btech11003_ex}

\begin{table}[!ht]
\centering
\input{./tables/ee18btech11003_ex.tex}
\caption{}
\label{table:ee18btech11003_ex}
\end{table}

\item  Find the transfer function using Mason's Gain Formula.
\renewcommand{\thefigure}{\theenumi.\arabic{figure}}
%
\\
\solution From \eqref{eq:ee18btech11003_mason} and \ref{table:ee18btech11003_ex},
\begin{align}
T(s)&=\frac{P_1 \Delta_1+P_2 \Delta_2}{\Delta}
\\
&=\frac{1 +\frac{1}{s^2}}{1-(-s-1-\frac{1}{s}-\frac{1}{s^2})}
\\
&=\frac{s^2+1}{s^3+2s^2+s+1}
\end{align}
%
after simplification.
\renewcommand{\thefigure}{\theenumi}
\item State the Equivalent Matrix Form of  Masons Gain formula 
\\
\solution Mason's rule can be stated in a simple matrix form. Assume T is  the transient matrix of the graph where \[t_{nm}\ =[T_{nm}]\] is sum transmittance of branches from node m toward node n. Then, the gain from node m to node n of the graph is equal to \[u_{nm}\ =[U_{nm}]\] where,
\begin{align}
    U=(I-T)^-1
\end{align}
and I is the identity matrix.
\item Find the Transfer Function using the matrix equivalent form of Masons Gain Formula
\\
\solution The transient matrix for \ref{fig:ee18btech11003_signal_flow}  is 
\begin{align}
\vec{T} = \myvec{0 & 1 & 0 & 0 & 0 \ \\
0 & 0 & s+1/s & 0 & 0   \\
 0 & 0 & 0 & 1 & 0 \\
 -1 & 0 & 0 & 0 &1/s \\
-1 & 0 & 0 & 0 & 0 }
\end{align}
\begin{align}
\vec{I} = \myvec{1 & 0 & 0 & 0 & 0 \ \\
0 & 1 & 0 & 0 & 0   \\
 0 & 0 & 1 & 0 & 0 \\
 0 & 0 & 0 & 1 & 0 \\
0 & 0 & 0 & 0 & 1 }
\end{align}
\begin{align}
\vec{I}-\vec{T} = \myvec{1 & -1 & 0 & 0 & 0 \ \\
0 & 1 & -s-1/s & 0 & 0   \\
 0 & 0 & 1 & -1 & 0 \\
 1 & 0 & 0 & 1 &-1/s \\
1 & 0 & 0 & 0 & 1 }
\end{align}
\begin{align}
    U =(\vec{I}-\vec{T})^-1
\end{align}
our transfer function is $U_0_4$
\begin{align}
U_0_4=C_4_0/|\vec{I}-\vec{T}|
\label{eq:transfer_function}
\end{align}
\begin{align}
    C_4_0 &=\mydet{
-1&0&0&0 
\\1 &-s-1/s&0&0 
\\0&1&-1&0
\\0&0&1&-1/s}
\end{align}
\begin{align}
    C_4_0 &=-1\mydet{
-s-1/s&0&0\\1&-1&0 
\\0&1&-1/s}
\end{align}
\begin{align}
    C_4_0 &=(s+1/s)(1/s)=\frac{s^2+1}{s^2}
    \label{eq:C40}
\end{align}
\begin{align}
|\vec{I}-\vec{T}|=1\mydet{1&-s-1/s&0&0 \\0&1&-1&0 \\0&0&1&-1/s\\0&0&0&1}  \\ +1\mydet{0&-s-1/s&0&0 \\0&1&-1&0 \\1&0&1&-1/s\\1&0&0&1}
\end{align}
\begin{align}
|\vec{I}-\vec{T}|= 1\mydet{1&-1&0\\0&1&-1/s\\0&0&1}+(s+1/s)\mydet{0&-1&0\\0&1&-1/s\\0&0&1}\ \\+(s+1/s)\mydet{0&-1&0\\1&1&-1/s\\1&0&1}   
\end{align}
\begin{align}
|\vec{I}-\vec{T}|=1+0+(s+1/s)(1+1/s)
\end{align}
\begin{align}
|\vec{I}-\vec{T}|=s+2+ 1/s +1/s^2
\end{align}
\begin{align}
 |\vec{I}-\vec{T}|=\frac{s^3+2s^2+s+1}{s^2}
 \label{eq:det}
\end{align}
from \eqref{eq:transfer_function},\eqref{eq:C40},\eqref{eq:det} we get
\begin{align}
    T(s)=\frac{s^2+1}{s^3+2s^2+s+1}
\end{align}
\\
\item Write a program to compute Mason's gain formula, given the branch nodes and gains for each path.
\\
\solution below code finds Masons Gain 
\begin{lstlisting}
codes/MasonsGain.py
\end{lstlisting}
\end{enumerate}

\caption{}
\label{table:ee18btech11003}
\end{table}
\item List the parameters in Table \ref{table:ee18btech11003}
for Fig. \ref{fig:ee18btech11003_signal_flow}.
\\
\solution The parameters are available in Table \ref{table:ee18btech11003_ex}

\begin{table}[!ht]
\centering
\input{./tables/ee18btech11003_ex.tex}
\caption{}
\label{table:ee18btech11003_ex}
\end{table}

\item  Find the transfer function using Mason's Gain Formula.
\renewcommand{\thefigure}{\theenumi.\arabic{figure}}
%
\\
\solution From \eqref{eq:ee18btech11003_mason} and \ref{table:ee18btech11003_ex},
\begin{align}
T(s)&=\frac{P_1 \Delta_1+P_2 \Delta_2}{\Delta}
\\
&=\frac{1 +\frac{1}{s^2}}{1-(-s-1-\frac{1}{s}-\frac{1}{s^2})}
\\
&=\frac{s^2+1}{s^3+2s^2+s+1}
\end{align}
%
after simplification.
\renewcommand{\thefigure}{\theenumi}
\item State the Equivalent Matrix Form of  Masons Gain formula 
\\
\solution Mason's rule can be stated in a simple matrix form. Assume T is  the transient matrix of the graph where \[t_{nm}\ =[T_{nm}]\] is sum transmittance of branches from node m toward node n. Then, the gain from node m to node n of the graph is equal to \[u_{nm}\ =[U_{nm}]\] where,
\begin{align}
    U=(I-T)^-1
\end{align}
and I is the identity matrix.
\item Find the Transfer Function using the matrix equivalent form of Masons Gain Formula
\\
\solution The transient matrix for \ref{fig:ee18btech11003_signal_flow}  is 
\begin{align}
\vec{T} = \myvec{0 & 1 & 0 & 0 & 0 \ \\
0 & 0 & s+1/s & 0 & 0   \\
 0 & 0 & 0 & 1 & 0 \\
 -1 & 0 & 0 & 0 &1/s \\
-1 & 0 & 0 & 0 & 0 }
\end{align}
\begin{align}
\vec{I} = \myvec{1 & 0 & 0 & 0 & 0 \ \\
0 & 1 & 0 & 0 & 0   \\
 0 & 0 & 1 & 0 & 0 \\
 0 & 0 & 0 & 1 & 0 \\
0 & 0 & 0 & 0 & 1 }
\end{align}
\begin{align}
\vec{I}-\vec{T} = \myvec{1 & -1 & 0 & 0 & 0 \ \\
0 & 1 & -s-1/s & 0 & 0   \\
 0 & 0 & 1 & -1 & 0 \\
 1 & 0 & 0 & 1 &-1/s \\
1 & 0 & 0 & 0 & 1 }
\end{align}
\begin{align}
    U =(\vec{I}-\vec{T})^-1
\end{align}
our transfer function is $U_0_4$
\begin{align}
U_0_4=C_4_0/|\vec{I}-\vec{T}|
\label{eq:transfer_function}
\end{align}
\begin{align}
    C_4_0 &=\mydet{
-1&0&0&0 
\\1 &-s-1/s&0&0 
\\0&1&-1&0
\\0&0&1&-1/s}
\end{align}
\begin{align}
    C_4_0 &=-1\mydet{
-s-1/s&0&0\\1&-1&0 
\\0&1&-1/s}
\end{align}
\begin{align}
    C_4_0 &=(s+1/s)(1/s)=\frac{s^2+1}{s^2}
    \label{eq:C40}
\end{align}
\begin{align}
|\vec{I}-\vec{T}|=1\mydet{1&-s-1/s&0&0 \\0&1&-1&0 \\0&0&1&-1/s\\0&0&0&1}  \\ +1\mydet{0&-s-1/s&0&0 \\0&1&-1&0 \\1&0&1&-1/s\\1&0&0&1}
\end{align}
\begin{align}
|\vec{I}-\vec{T}|= 1\mydet{1&-1&0\\0&1&-1/s\\0&0&1}+(s+1/s)\mydet{0&-1&0\\0&1&-1/s\\0&0&1}\ \\+(s+1/s)\mydet{0&-1&0\\1&1&-1/s\\1&0&1}   
\end{align}
\begin{align}
|\vec{I}-\vec{T}|=1+0+(s+1/s)(1+1/s)
\end{align}
\begin{align}
|\vec{I}-\vec{T}|=s+2+ 1/s +1/s^2
\end{align}
\begin{align}
 |\vec{I}-\vec{T}|=\frac{s^3+2s^2+s+1}{s^2}
 \label{eq:det}
\end{align}
from \eqref{eq:transfer_function},\eqref{eq:C40},\eqref{eq:det} we get
\begin{align}
    T(s)=\frac{s^2+1}{s^3+2s^2+s+1}
\end{align}
\\
\item Write a program to compute Mason's gain formula, given the branch nodes and gains for each path.
\\
\solution below code finds Masons Gain 
\begin{lstlisting}
codes/MasonsGain.py
\end{lstlisting}
\end{enumerate}

\caption{}
\label{table:ee18btech11003}
\end{table}
\item List the parameters in Table \ref{table:ee18btech11003}
for Fig. \ref{fig:ee18btech11003_signal_flow}.
\\
\solution The parameters are available in Table \ref{table:ee18btech11003_ex}

\begin{table}[!ht]
\centering
\input{./tables/ee18btech11003_ex.tex}
\caption{}
\label{table:ee18btech11003_ex}
\end{table}

\item  Find the transfer function using Mason's Gain Formula.
\renewcommand{\thefigure}{\theenumi.\arabic{figure}}
%
\\
\solution From \eqref{eq:ee18btech11003_mason} and \ref{table:ee18btech11003_ex},
\begin{align}
T(s)&=\frac{P_1 \Delta_1+P_2 \Delta_2}{\Delta}
\\
&=\frac{1 +\frac{1}{s^2}}{1-(-s-1-\frac{1}{s}-\frac{1}{s^2})}
\\
&=\frac{s^2+1}{s^3+2s^2+s+1}
\end{align}
%
after simplification.
\renewcommand{\thefigure}{\theenumi}
\item Write a program to compute Mason's gain formula, given the branch nodes and gains for each path.
\\
\solution The following code computes Masons gain Formula 
%
\begin{lstlisting}
codes/MasonsGain.py
\end{lstlisting}
\end{enumerate}